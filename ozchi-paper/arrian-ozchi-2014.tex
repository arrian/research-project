\documentclass{sig-alternate}

\begin{document}
\conferenceinfo{OZCHI}{'14, Dec 2-5, 2014, Sydney, Australia}

\title{Visualising a Live Coding Arts Process}

\numberofauthors{3} 
\author{
\alignauthor Anon1\\
       \affaddr{-}\\
       \affaddr{-}\\
       \email{-}
\alignauthor Anon2\\
       \affaddr{-}\\
       \affaddr{-}\\
       \email{-}
\alignauthor Anon3\\
       \affaddr{-}\\
       \affaddr{-}\\
       \email{-}
}

\maketitle
\begin{abstract}

  We have investigated software visualisations as a means to
  communicate the programming process of ``live coding'' computer
  music performances. Following a field study at a festival of the
  contemporary arts, two sets of complementary, interaction-driven
  ``aesthetic'' and ``didactic'' visualisations were incorporated into a
  live coding system. A formal audience experiment was undertaken to
  evaluate the visualisations' contributions to the audience
  experience, with emphasis on two (self-reported) experiential
  dimensions of ``understanding'' and ``enjoyment''. The reactions of
  the live coding performer (the programmer) were also evaluated in a
  video-cued-recall interview.

  A majority of the audience in the formal experiment reported that
  both sets of software visualisations enhanced their ``enjoyment''.
  Reported audience ``understanding'' was more problematic. Although
  more of the audience reported that the ``didactic'' (rather than the
  ``aesthetic'') visualisations helped them to ``understand'' the
  process, the reported levels of ``understanding'' fluctuated during
  the course of the performances in a way that did not seem to
  preference one visualisation type over the other. Evaluation of the
  reactions of the live coder were notably different to that of the
  audience. This observation, and our study overall, motivate an
  ongoing challenge to develop live coding visualisations which better
  align the mental models of performer and audience to enhance the
  experiences of all.

\end{abstract}

\category{H.5}{Information Interfaces and Presentation}{Miscellaneous}

\terms{Experimentation, Design}

\keywords{Live coding, software visualisation}

\section{Introduction}

% this first sentence could possibly be the abstract - along with the
% rhetorical question of ``what, exactly, are we showing?''

``Show us your screens\ldots Code should be seen as well as heard'',
declares the draft manifesto of ``TOPLAP''~\cite{Toplap}, an
international organisation devoted to the artistic performance
practice of ``live coding''. In live coding, computer code is written
in front of a live audience to generate music and visuals in real
time. The ``show us your screens'' rhetoric underscores the need for
authenticity to distinguish this artform from similar (but non-live)
computational arts practices.

But what is the benefit of the live coder showing their screen? In a
live coding performance, non-expert live coding audience members spend
much of their time staring at raw (usually text-based) computer code.
Until now, little empirical study has been undertaken to gauge an
audience's understanding of that computer code and whether, from an
audience perspective, code really should be ``seen as well as heard''.

Traditional approaches to source code visualisation (\cite[see][for a
review]{Novais2013}) often focus on structure of the source code (e.g.
visualising complex object/class relationships) rather than the
\emph{process} of programming. In a process-oriented activity such as
live coding, different code visualisation techniques are
necessary~\cite{McLean2010b,Magnusson2013}. However, these academic
treatments of code visualisation in live coding adopt a survey-based
approach, and the techniques discussed have not been subject to
empirical evaluation.

In this paper, we examine the audience's experience of the displayed
code during live coding performances and to see whether code-driven
visualisations might improve both the audience enjoyment and the
audience understanding of these performances. This exploration takes
place initially through the results of an exploratory field study at a
contemporary arts festival, and subsequently through a lab-based
followup user study.

\section{Exploratory Field Study}

After a livecoding performance at the [name withheld for blind
peer-review] arts festival in [place and date withheld], audience
members were asked to fill out a survey regarding their perception of
and response to the projection of the computer code during the
performance. Each audience member was asked to indicate which of a
number of curves/trajectories best represented their \emph{enjoyment}
and \emph{understanding} of the performer's actions in typing the code
through the performance. These trajectories allowed for ``high'',
``medium'', and ``low'' levels of enjoyment/understanding for the
(self-determined) ``beginning'', ``middle'' and ``end'' of the
performance. Other survey questions addressed their sense of
``liveness'' of the performance \cite{Auslander} and whether the
projected code was confusing.

\subsection{Field Study Results}

Of the thirteen survey responses received, six audience members showed
a high level of enjoyment throughout the whole performance, while the
remaining seven responses showed alternating levels of enjoyment. No
audience members indicated a low level of enjoyment throughout the
performance.

Only two of the thirteen respondents indicated that they understood
the relationship between the code projections and the music throughout
the performance. Three of the six respondents who reported a high
level of enjoyment throughout the performance also indicated an
increase in understanding (from low to high) as the performance
progressed, although a Chi-square analysis revealed no significant
relationship between enjoyment and understanding due to the small
sample size. Nine of the thirteen respondents stated that the code
projection provided a sense of liveness to the performance and the
remainder stated that viewing the code had no effect on their sense of
liveness. Four respondents felt that the code projections were
confusing, five felt that they were not confusing, and four did not
answer the question.

Taken as a whole, the results of this small field study were
salutatory for the benefit of ``seeing as well as hearing'' code
during a live coding performance, especially as far as the general
public is concerned. The majority of the audience felt that they made
the performance seem more ``live''. However, a minority stated that
they found the projections confusing and only a very small number of
respondents claimed to have actually understood what the programmer
was doing. We were quite intrigued by the small cohort of respondents
whose understanding increased through the performance and whose
enjoyment remained high, and we wished to test whether augmenting code
projections with additional visualisations might increase the
understanding and enjoyment of the audience in live coding.

\section{Lab Study}

\begin{figure}
\centering
\epsfig{file=aesthetic-vis.eps, width=\columnwidth}
\caption{An example aesthetic visualisation (all
  figures best viewed in colour).}
\label{fig:aesthetic-visualisation}
\end{figure}

\begin{figure}
\centering
\epsfig{file=didactic-vis.eps, width=\columnwidth}
\caption{An example didactic visualisation.}
\label{fig:didactic-visualisation}
\end{figure}

Two sets of visualisations were developed based on the results of the
initial survey and the literature TODO in what way?. The first set
(the ``didactic'' visualisations) attempted to communicate the actions
of the programmer and the main software components that were created
and modified by the programmer over time (see
Figure~\ref{fig:didactic-visualisation}). The second set (the
``aesthetic'' visualisations) focussed on increasing audience
retention and user experience (TODO how?) by maximising aesthetic
appeal~\cite{Cawthon2007} reacting to musical changes triggered by the
programmer (see Figure~\ref{fig:aesthetic-visualisation}).

The ``didactic'' visualisations primarily focussed on the relationship
between the active software processes and their behaviour. They
prominently display the \emph{names} of the active functions
together with indications of the number of functions running and their
callback times. Bright colours and solid shapes were used to ensure
constant visibility and to communicate the intention of the underlying
code. Overall, up-to four visualisations were presented with each
introduced depending on the number of active functions. It was
hypothesised that taking a didactic approach to visualising the
live-coding process would result in enhanced audience understanding,
and a reduction in audience confusion through the performance.

The set of ``aesthetic'' visualisations focussed less on the
programmatic aspects of the live coding performance, rather intending
to provide additional visual interest to the performance and thereby
prolonging attention. The generation of shapes was driven by
instrument volume and synchronised with the musical beat. Compared
with the ``didactic'' set, more variety was used in visual structure
and colour. Once again, four visualisations were presented, varying
the visualisation based on the number of active software functions,
but these visualisations were unlabelled and they moved and interacted
with each other over the entire projected scene. It was predicted that
focussing on the aesthetic nature of the visualisations would assist
in audience retention and result in a consistency of interest through
the performance.
% \cite{Cawthon2007} (ALSO CITE MClEAN?) (HOW EXACTLY DID YOU MAXIMISE AESTHETIC APPEAL? HOW WERE THE AESTHETIC VISUALISATIONS DIVEN? WERE THEY IN SYNCH WITH THE METRONOME?)

\subsection{Lab Study Experimental Design}

% todo crossed design, or whatever. between-groups. latin square?

Our visualisations were presented to the audiences of two live coding
performances. Each performance was run as two ten-minute, improvised
computer music pieces with each piece demonstrating one of the two
visualisation types. The experiment was run twice, with
non-overlapping audience members. The order of presentations of the
visualisation types was swapped between performances. Audience members
were recruited on the basis of being treated to a computer music
performance. In our experimental design we considered, but decided
against, the inclusion of a third condition of ``code with no
visualisation'' on the basis of the resulting experimental design
making it difficult to recruit audience participants and causing
audience and performer fatigue (particularly when taking
order-balancing into consideration).

Over the course of each performance, the audience members completed a
survey consisting of four sections. These included demographic
information, their opinion of the first piece, their opinion of the
second piece and some overall questions. Questions regarding each
piece primarily focussed on reporting levels of ``enjoyment'' and
``understanding'' related to the visualisation while the final section
solicited suggestions for improvements to the visualisation.
% The experiment and survey design were endorsed by Ethics Protocol (details omitted for blind review).

Subsequent to the audience experiment a video-cued-recall interview
was conducted with the live coder. One performance was examined
critically and comparisons were drawn between the live coder's
statements and the results of the audience surveys.

\subsection{Lab Study Results}

A total of 41 audience survey responses were received with a split of
19 and 22 participants in each of the performances. Of this survey
population, $66\%$ were male, $76\%$ were aged between 18 and 32 and
$78\%$ of the participants had never seen a live coding performance
before.

Audience ``enjoyment'' and ``understanding'' were evaluated for the
two sets of visualisations as described below. A significance level of
$0.05$ was used for the Chi-squared analysis.

\subsubsection{Enjoyment}

\begin{figure}
  \centering \epsfig{file=aesthetic-enjoyment-final.eps, width=\columnwidth}
  \caption{Audience reported enjoyment level during the beginning,
    middle and end of the performance with the ``aesthetic''
    condition. For this and all subsequent graphs of this type, line
    width indicates proportion of the audience reporting high, medium
    or low and link colour between the three performance stages
    indicates the reported level at the \emph{beginning} of the
    performance.}
\label{fig:aesthetic-enjoyment}
\end{figure}

\begin{figure}
\centering
\epsfig{file=didactic-enjoyment-final.eps, width=\columnwidth}
\caption{Audience reported enjoyment during the beginning, middle and
  end of the performance with the ``didactic'' condition.}
\label{fig:didactic-enjoyment}
\end{figure}

Overall, a the majority of the participants stated that both
visualisations had a positive effect on their ``enjoyment'' of the
performance: $76\%$ stated that the ``aesthetic'' visualisations
helped their ``enjoyment'' and $56\%$ stated that the ``didactic''
visualisations helped their ``enjoyment''. No significant difference
between the visualisation types on ``enjoyment'' was found
($\chi^2=3.7733,df=2,p=0.1516$).

Participants were asked to rate their ``enjoyment'' during the
(self-reported) ``beginning'', ``middle'' and ``end'' of the
performances (see Figure~\ref{fig:aesthetic-enjoyment} and
Figure~\ref{fig:didactic-enjoyment}). During the ``didactic''
performance, $15\%$ of the audience stated that their ``enjoyment''
\emph{increased} from the beginning of the performance and was
steady thereafter. By contrast, $24\%$ of the audience
reported this pattern of ``enjoyment'' during the ``aesthetic''
performance. Approximately $30\%$ of the audience of all
(``aesthetic'' and ``didactic'') performances stated that their
enjoyment remained steady throughout.

\subsubsection{Understanding}

\begin{figure}
\centering
\epsfig{file=aesthetic-understanding-final.eps, width=\columnwidth}
\caption{Audience reported understanding during the beginning, middle
  and end of the performance with the ``aesthetic'' condition.}
\label{fig:aesthetic-understanding}
\end{figure}

\begin{figure}
\centering
\epsfig{file=didactic-understanding-final.eps, width=\columnwidth}
\caption{Audience reported understanding during the beginning, middle
  and end of the performance with the ``didactic'' condition.}
\label{fig:didactic-understanding}
\end{figure}

In response to a specific survey question, $37\%$ of participants
stated that, overall, the ``didactic'' visualisations \emph{helped
  them to understand the code} whereas, by contrast, $12\%$
participants stated that, overall, the ``aesthetic'' visualisations
\emph{helped them to understand the code}. This was a significant
difference between these responses ($\chi^2=7.1986,df=2,p=0.02734$).

Participants were asked to rate their ``understanding'' during the
(self-reported) ``beginning'', ``middle'' and ``end'' of the
performance (see Figure~\ref{fig:aesthetic-understanding} and
Figure~\ref{fig:didactic-understanding}). During the ``didactic'' and
``aesthetic'' performances, $49\%$ and $44\%$ of the participants
stated, respectively, that their ``understanding'' \emph{remained
  the same} throughout the performances. During the ``didactic''
performance, $10\%$ of the audience reported a level of
``understanding'' that \emph{trended downwards} (eg. high to low)
compared to $20\%$ of the audience during the aesthetic performance.
However, this reported advantage of the ``didactic'' visualisations
was offset by the reported audience ``understanding'' at the beginning
of the performance where $44\%$ indicated a low ``understanding'' with
the ``didactic'' visualisations compared to only $30\%$ with the
``aesthetic'' visualisations. Overall, the figures for
``understanding'' are complex and reported levels of ``understanding''
fluctuated during the performances. Dramatically,
Figure~\ref{fig:didactic-understanding} shows that a very small
proportion of the audience reported ``high understanding'' during the
``middle'' of the performances. One interpretation of this result
might be that it took audience members some time to work out what the
``didactic'' visualisations were actually showing, and that this
conflicted with the first impressions of what some audience members
(hence the decrease in levels of ``understanding'' from beginning to
middle). However, once they finally understood the graphics some
audience members were then able to better ``understand'' the
live-coding performance.

\subsection{Discussion}

Overall ``enjoyment'' of the visualisations was high. This was the
case for both the ``aesthetic'' and ``didactic'' visualisations.
Reported ``enjoyment'' of the ``aesthetic'' visualisations was higher
than for the ``didactic'' visualisations but the trends across
Figures~\ref{fig:aesthetic-enjoyment} and~\ref{fig:didactic-enjoyment}
are complex.

As discussed above, the small number of ``high'' responses for
``understanding'' during the ``middle'' of the ``didactic''
performances, and the decreasing trend from ``high'' to ``middle''
``understanding'' from ``beginning'' to ``middle'' of the performances
perhaps indicates a higher cognitive load for understanding the
``didactic'' visualisations themselves. In fact, features of the
``didactic'' visualisation were reported to confuse some members of
the audience. One audience member even stated that they ``found them
distracting'' and that they ``preferred just to read the code''.

The video-cued-recall interview indicated that the experience of the
visualisations of the live coder and the audience was fundamentally
different. Whereas many members of the audience reported that they
drifted between focussing on the music, focussing on the
visualisations and focussing on the code, the live coder reported that
their focus was purely on the code and the music, rarely drifting. In
one particular section of the interview, the live coder stated: ``I
definitely wasn't paying attention to them [the visualisations] on the
day. In fact I tune them out as best I can because I am just trying to
focus on the code''. By contrast, one audience member stated that
``you could see the code being written and the visualisations helped
to show when a piece of code started working''. Another audience
member stated that ``the visualisations were interesting but
distracting''. By contrast, when asked if the visualisations were
distracting the live coder stated: ``Ah, no. In general I'm just so
focussed on the code''.

\section{Conclusion}

In the first systematic study of its type, we have identified an
opportunity for real-time code visualisations to help improve the
audience experience of a live coding computer music performance. With
few exceptions, our initial survey of a live coding performance at an
arts festival revealed a generally low to medium level of audience
``understanding'' throughout that performance (although almost half
the survey respondents indicated a high level of ``enjoyment''
throughout).

In an audience experiment, our comparison of two prototype code
visualisations indicated that both visualisations seemed to ``help''
with ``enjoyment''. Significantly more audience members reported that
our ``didactic'' visualisations ``helped'' with ``understanding'' but
overall trends for both ``enjoyment'' and ``understanding'' throughout
the performances were complex. There are indications of a higher
cognitive load for the ``didactic'' visualisations than the
``aesthetic'' visualisations and this may have influenced audience
responses to them.

In a future extension of this work, design lessons from both
visualisation types could be combined together to produce live coding
driven visualisations which targeted both the aesthetics as well as a
greater understanding of the live coding process. These visualisations
could then be compared with the baseline ``no-visualisation''
condition in an audience experiment. TODO also vary the visuals over
the course of a performance?

Over 60 years ago, the media theorist Marshall McLuhan stated that
``The business of art is no longer the communication of thoughts or
feelings which are to be conceptually ordered, but a direct
participation in an experience. The whole tendency of modern
communication... is towards participation in a process, rather than
apprehension of concepts.'' \cite{McLuhan} Our hope is that future
development of the research described in this paper will bring
audiences into the \emph{process of live coding} in such a way that
they might feel to be active (virtual) collaborators of a
highly-skilled live-coding artist.

\bibliographystyle{abbrv}
\bibliography{sigproc}  % sigproc.bib % ACM needs 'a single self-contained file'!

\end{document}
