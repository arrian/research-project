\documentclass{sig-alternate}

\begin{document}
\conferenceinfo{OZCHI}{'14, Dec 2-5, 2014, Sydney, Australia}

\title{Visualising a Live Coding Arts Process}

\numberofauthors{3} 
\author{
\alignauthor Anon1\\
       \affaddr{-}\\
       \affaddr{-}\\
       \email{-}
\alignauthor Anon2\\
       \affaddr{-}\\
       \affaddr{-}\\
       \email{-}
\alignauthor Anon3\\
       \affaddr{-}\\
       \affaddr{-}\\
       \email{-}
}

\maketitle
\begin{abstract}
  This paper describes an empirical study of source code visualisation
  as a means to communicate the programming process in ``live coding''
  computer music performances. Following an exploratory field study
  conducted during a live coding performance at an arts festival, two
  different interaction-driven visualisation techniques were
  incorporated into a live coding system. We then performed a more
  controlled laboratory study to evaluate the visualisations' contributions
  to the audience experience, with emphasis on the (self-reported)
  experiential dimensions of \emph{understanding} and
  \emph{enjoyment}. Both software visualisation techniques enhanced
  audience enjoyment, while the effect on audience understanding was
  more complex. We conclude by suggesting how these visualisation
  techniques may be used to enhance the audience experience of live
  coding.
\end{abstract}

\category{H.5}{Information Interfaces and Presentation}{Miscellaneous}

\terms{Experimentation, Design}

\keywords{Software visualisation, live coding, musical performance}

\section{Introduction}

``Show us your screens\ldots Code should be seen as well as heard'',
declares the draft manifesto of ``TOPLAP''~\cite{Toplap}, an
international organisation devoted to the artistic performance
practice of ``live coding''. In live coding, computer code is written
in front of a live audience to generate music and visuals in real
time. The ``show us your screens'' rhetoric underscores the need for
authenticity to distinguish this artform from similar (but non-live)
computational arts practices.

But what is the benefit of the live coder showing their screen? In a
live coding performance, non-expert live coding audience members spend
much of their time staring at raw (usually text-based) computer code.
Until now, little empirical study has been undertaken to gauge an
audience's understanding of that computer code and whether, from an
audience perspective, code really should be ``seen as well as heard''.

Traditional approaches to source code visualisation
(see~\cite{Novais2013} for a review) often focus on structure of the
source code (e.g. visualising complex object/class relationships)
rather than the \emph{process} of programming. In a process-oriented
activity such as live coding, different code visualisation techniques
are necessary~\cite{McLean2010b,Magnusson2013}. However, these
academic treatments of code visualisation in live coding adopt a
survey-based approach, and the techniques discussed have not been
subject to empirical evaluation.

In this paper, we examine the audience's experience of the displayed
code during live coding performances and to see whether code-driven
visualisations might improve both the audience enjoyment and the
audience understanding of these performances. This exploration takes
place initially through the results of an exploratory field study at a
contemporary arts festival, and subsequently through a laboratory-based
follow-up user study.

\section{Exploratory Field Study}

After a live coding performance at the [name withheld for blind
peer-review] arts festival in [place and date withheld], audience
members were asked to fill out a survey regarding their perception of
and response to the projection of the computer code during the
performance. Each audience member was asked to indicate which of a
number of curves/trajectories best represented their \emph{enjoyment}
and \emph{understanding} of the performer's actions in typing the code
through the performance. These trajectories allowed for ``high'',
``medium'', and ``low'' levels of enjoyment/understanding for the
(self-determined) ``beginning'', ``middle'' and ``end'' of the
performance. Other survey questions addressed their sense of
``liveness'' of the performance (c.f.~\cite{Auslander}) and whether
the projected code was confusing.

\subsection{Field Study Results}

Of the thirteen survey responses received, six audience members showed
a high level of enjoyment throughout the whole performance, while the
remaining seven responses showed alternating levels of enjoyment. No
audience members indicated a low level of enjoyment throughout the
performance.

Only two of the thirteen respondents indicated that they understood
the relationship between the code projections and the music throughout
the performance. Three of the six respondents who reported a high
level of enjoyment throughout the performance also indicated an
increase in understanding (from low to high) as the performance
progressed, although a Chi-square analysis revealed no significant
relationship between enjoyment and understanding due to the small
sample size. Nine of the thirteen respondents stated that the code
projection provided a sense of liveness to the performance and the
remainder stated that viewing the code had no effect on their sense of
liveness. Four respondents felt that the code projections were
confusing, five felt that they were not confusing, and four did not
answer the question.

Taken as a whole, the results of this small field study were
salutatory towards the benefit of ``seeing as well as hearing'' code
during a live coding performance, especially as far as the general
public is concerned. The majority of the audience felt that they made
the performance seem more ``live''. However, a minority stated that
they found the projections confusing and only a very small number of
respondents claimed to have actually understood what the programmer
was doing. We were quite intrigued by the small cohort of respondents
whose understanding increased through the performance and whose
enjoyment remained high, and we wished to test whether augmenting code
projections with additional visualisations might increase the
understanding and enjoyment of the audience in live coding.

\section{Laboratory Study}

\begin{figure}
\centering
\epsfig{file=didactic-vis-overlay.eps, width=\columnwidth}
\caption{An example didactic visualisation (all
  figures best viewed in colour).}
\label{fig:didactic-visualisation}
\end{figure}

\begin{figure}
\centering
\epsfig{file=aesthetic-vis-overlay.eps, width=\columnwidth}
\caption{An example aesthetic visualisation.}
\label{fig:aesthetic-visualisation}
\end{figure}

A second laboratory study was conducted to test the impact of additional
visual feedback (beyond the raw source code) on audience understanding
and enjoyment in live coding. Music visualisation is an extremely rich
and open-ended task, so to guide the development of the visualisations
for our laboratory study, we used the concepts of understanding and enjoyment
from the initial survey to develop two new code visualisations: a
\emph{didactic} one and an \emph{aesthetic} one.

The didactic visualisation (shown in
Figure~\ref{fig:didactic-visualisation}) attempted to communicate
\emph{information} about the actions of the programmer, prominently
displaying the \emph{names} of the active (source code) functions and
the ``time until next execution'' for each function (which is
particularly relevant in a time-sensitive programming context such as
music making). Bright colours and solid shapes were used to ensure
constant visibility and to communicate the intention of the underlying
code. The didactic visualisations proceeded through four stages, with
phase changes made depending on the number of active functions
(instruments).

The aesthetic visualisation technique, on the other hand, was designed
to react to the programmer's activity in a more abstract way, to
maximise aesthetic appeal~\cite{Cawthon2007} and to engage the
audience's interest. Although still based on the source code and the
livecoder's edits, the generation of shapes was driven by instrument
volume and synchronised with the musical beat. The emphasis for the
aesthetic visualisation was on the artistic appeal of the visuals (see
Figure~\ref{fig:aesthetic-visualisation}), including more variety in
visual structure and colour. As in the didactic condition, the
aesthetic visualisations proceeded through four stages, based on the
number of active functions (instruments), but these visuals had no
textual labels and they moved and interacted with each other over the
entire projected scene.

Our hypothesis was the didactic visualisation approach would result in
enhanced audience understanding, and a reduction in audience confusion
through the performance. In contrast, we predicted that the aesthetic
visualisations would positively influence audience enjoyment, both
overall and over the course of the performance.

\subsection{Laboratory Study Experimental Design}

To assess the impact of these two visualisation techniques on audience
understanding and enjoyment, we conducted a laboratory study. Two independent
audiences ($N=19+22=41$) recruited through an on-campus advertisement
each watched a live coder perform two ten-minute ``sets'': one
accompanied by the didactic visuals, and one with the aesthetic. The
order of presentation of the two visual conditions was swapped between
the groups. The improvisational nature of a live coding performance
makes ``controlled'' experiments difficult, but the live coding artist
attempted (as much as possible) to do the same two performances for
each group.

Over the course of these performances, each audience member completed
a survey consisting of four sections: demographic information, their
opinion of the first piece, their opinion of the second piece and
questions about the performance overall. Similar to the first field
trial, the questionnaire primarily focussed on self-reported levels of
``enjoyment'' and ``understanding'' related to the visualisations
specifically and also to the performance more generally. There was
also a free-form question for suggested improvements to the
visualisations.

After the laboratory study performance, a
video-cued-recall~\cite{Suchman:1992tk} interview was conducted with
the live coder using a video of the performance.

\subsection{Laboratory Study Results}

Of the $41$ audience participants $66\%$ were male, $76\%$ were aged
between 18 and 32 and $78\%$ of the participants had never seen a live
coding performance before.

The audience-reported enjoyment and understanding responses from the
questionnaire were evaluated for the two visualisation conditions as
described below. A significance level of $0.05$ was used for the
Chi-squared analysis.

\subsubsection{Enjoyment}

\begin{figure}
\centering
\epsfig{file=didactic-enjoyment-final.eps, width=\columnwidth}
\caption{Audience reported enjoyment during the beginning, middle and
  end of the performance for the \textbf{didactic} condition.}
\label{fig:didactic-enjoyment}
\end{figure}

\begin{figure}
  \centering \epsfig{file=aesthetic-enjoyment-final.eps, width=\columnwidth}
  \caption{Audience-reported enjoyment level during the beginning,
    middle and end of the performance for the \textbf{aesthetic} condition.
    Line width at each stage indicates proportion of the audience
    reporting high, medium or low enjoyment, and line colour is
    determined by the enjoyment level at the \emph{beginning} of the
    performance.}
\label{fig:aesthetic-enjoyment}
\end{figure}

Overall, the majority of the participants reported that both
visualisation conditions had a positive effect on their
\textbf{enjoyment} of the performance: $76\%$ stated that the
aesthetic visualisations improved their enjoyment and $56\%$ stated
that the didactic visualisations improved their enjoyment. No
significant difference between the visualisation types regarding 
enjoyment was found ($\chi^2=3.7733,df=2,p=0.1516$).

Participants were asked to rate their enjoyment during the
(self-determined) ``beginning'', ``middle'' and ``end'' of the
performances (see Figure~\ref{fig:aesthetic-enjoyment} and
Figure~\ref{fig:didactic-enjoyment}). During the didactic
performance, $15\%$ of the audience stated that their enjoyment
\emph{increased} from the beginning of the performance and was
steady thereafter. By contrast, $24\%$ of the audience
reported this pattern of enjoyment during the aesthetic
performance. Approximately $30\%$ of the audience of all
(aesthetic and didactic) performances stated that their
enjoyment remained steady throughout.

\subsubsection{Understanding}

\begin{figure}
\centering
\epsfig{file=didactic-understanding-final.eps, width=\columnwidth}
\caption{Audience reported understanding during the beginning, middle
  and end of the performance for the \textbf{didactic} condition.}
\label{fig:didactic-understanding}
\end{figure}

\begin{figure}
\centering
\epsfig{file=aesthetic-understanding-final.eps, width=\columnwidth}
\caption{Audience reported understanding during the beginning, middle
  and end of the performance for the \textbf{aesthetic} condition.}
\label{fig:aesthetic-understanding}
\end{figure}

In response to a specific survey question, $37\%$ of participants
stated that overall, the didactic visualisations ``helped them to
\textbf{understand} the code'', compared to $12\%$ of participants for the
aesthetic visualisations. This was a significant difference between
the visualisation conditions ($\chi^2=7.1986,df=2,p=0.02734$).

Again, participants were asked to rate their understanding during the
(self-reported) ``beginning'', ``middle'' and ``end'' of the
performance (see Figure~\ref{fig:aesthetic-understanding} and
Figure~\ref{fig:didactic-understanding}). During the didactic and
aesthetic performances, $49\%$ and $44\%$ respectively of the
participants stated that their understanding \emph{remained the same}
throughout the performances. During the didactic performance, $10\%$
of the audience reported a level of understanding that \emph{trended
  downwards} (eg. high to low) compared to $20\%$ of the audience
during the aesthetic performance. However, this reported advantage of
the didactic visualisations was offset by the reported audience
understanding at the beginning of the performance where $44\%$
indicated a low understanding with the didactic visualisations
compared to only $30\%$ with the aesthetic visualisations. 

Overall, the questionnaire results for audience understanding are
complex, and reported levels of understanding fluctuated during the
performances. Dramatically, Figure~\ref{fig:didactic-understanding}
shows that a very small proportion of the audience reported high
understanding during the middle of the performances. One
interpretation of this result might be that it took audience members
some time to work out what the didactic visualisations were actually
showing, and that this conflicted with the first impressions of what
some audience members (hence the decrease in levels of understanding
from beginning to middle). However, once they finally understood the
graphics some audience members were then able to better understand the
live-coding performance.

\subsection{Discussion}

The overall enjoyment of the visualisations was high, for both the
aesthetic and didactic visualisations. Reported enjoyment of the
aesthetic visualisations was higher than for the didactic
visualisations but the trends across
Figures~\ref{fig:aesthetic-enjoyment} and~\ref{fig:didactic-enjoyment}
are complex.

As discussed above, the small number of high responses for
understanding during the middle of the didactic performances, and the
decreasing trend from high to middle level understanding from
beginning to middle of the performances perhaps indicates a higher
cognitive load for understanding the didactic visualisations
themselves. In fact, features of the didactic visualisation were
reported to confuse some members of the audience, despite their stated
aim of \emph{assisting} audience understanding. One audience member
even stated that they ``found them distracting'' and that they
``preferred just to read the code''.

The video-cued-recall interview indicated that the experience of the
visualisations of the live coder and the audience was fundamentally
different. While many members of the audience reported that they
drifted between focussing on the music, focussing on the
visualisations and focussing on the code, the live coder reported that
their focus was purely on the code and the music, rarely drifting. In
one particular section of the interview, the live coder stated: ``I
definitely wasn't paying attention to them [the visualisations] on the
day. In fact I tune them out as best I can because I am just trying to
focus on the code''. By contrast, one audience member stated that
``you could see the code being written and the visualisations helped
to show when a piece of code started working''. Another audience
member stated that ``the visualisations were interesting but
distracting''. When asked if the visualisations were distracting the
live coder stated: ``Ah, no. In general I'm just so focussed on the
code''.

\section{Conclusion}

In this first empirical study of audience perception of code
visualisation in live coding, we have identified an opportunity for
real-time code visualisations to help improve the audience experience
of a live coding computer music performance. With few exceptions, our
initial survey of a live coding performance at an arts festival
revealed a generally low to medium level of audience understanding
throughout that performance (although almost half the survey
respondents indicated a high level of enjoyment throughout).

In a subsequent laboratory study, a comparison of two prototype code
visualisations indicated that both visualisations seemed to help with
enjoyment. Significantly more audience members reported that our
didactic visualisations helped with understanding but overall trends
for both enjoyment and understanding throughout the performances were
complex. There are indications of a higher cognitive load for the
didactic visualisations than the aesthetic visualisations and this may
have influenced audience responses to them.

In a future extension of this work, design lessons from both
visualisation types could be combined together to produce live coding
driven visualisations which targeted both the aesthetics as well as a
greater understanding of the live coding process. These visualisations
could then be compared with the baseline ``no visualisation''
condition in an audience experiment. There are also opportunities to
vary the nature of the visualisations over the course of a
performance.

Over 60 years ago, the media theorist Marshall McLuhan stated that
``The business of art is no longer the communication of thoughts or
feelings which are to be conceptually ordered, but a direct
participation in an experience. The whole tendency of modern
communication\ldots is towards participation in a process, rather than
apprehension of concepts.''~\cite{McLuhan} Our hope is that future
developments in visualisations for live coding may bring audiences
further into the \emph{process} of a highly-skilled live coding
artist.

\bibliographystyle{abbrv}
\bibliography{sigproc}  % sigproc.bib % ACM needs 'a single self-contained file'!

\end{document}
