
\section*{Interviewee 1}
\textit{Question 1: What did you ``understand'' about what was going on with the code being projected? In particular, what did you understand about the relationship between the code and the music?}

The code sets up a set of nested loops which are then modified by the composer in real time. This immediately leads to the danger of repetitive loops. It may be useful to have rhythms of 4 or 8 bar repetition as in Africa. Actually, this musical form lends itself to that type of rhythm and music. As soon as an organ comes in I am reminded of Mike Oldfield and am anxious that someone will say “slightly distorted guitar”. In jazz one improvises on standards so there is a very strong form (AABA etc) which, in general, is respected allowing the audience to deduce where they are in the piece. I had the feeling that the present way the code is used limited the music

\textit{Question 2: What would you like to understand more about the code in order to enjoy the performance more?}

Yes, I think that if the audience were told what was happening or the ideas behind the constructs then I would be happier. Compared to jazz it is not note by note improvisation so an explanation of the limits and advantages would be useful.

\section*{Interviewee 2}

\textit{Question 1: What did you ``understand'' about what was going on with the code being projected? In particular, what did you understand about the relationship between the code and the music?}

In the beginning, I could tell from the silence and the live coding that it was being build, and sound by sound line by line was being added to as the piece grew.  When [the live coder] went back in the code to change beats or melodies, I could tell something was being changed but wasn't tracking what or how.

\textit{Question 2: What would you like to understand more about the code in order to enjoy the performance more?}

I feel like I already understood a rudimentary amount [...] which was enough to enjoy it.  I feel like if I had more knowledge about code I would focus on that to the detriment of the music; and if I knew more about music then I may have focused on that to the detriment of my attention on the code.  Considering my education, if I had not had the exposure to code and music through [...], then some basic rudimentary knowledge of code would have been good.

\section*{Interviewee 3}

\textit{Question 1: What did you ``understand'' about what was going on with the code being projected? In particular, what did you understand about the relationship between the code and the music?}

I understood that the music was being made from scratch and this was evident in the long silence before any sound is heard. I understand what sounds are being made based on the code names and that some of the numbers represent timing, volume and pitch.  I still don't quite understand when the code is ``ready'' and starts working to make music. It has something to do with the highlighting the text, but that also confuses me. (I understand that most of the music is stored in the program as ``sound bites'' of real instruments, but sounds can also be made from scratch as mathematical wave functions [...]). Sometimes the coder scrolls up and down the screen to much and I get lost, I don't have a big picture of what all the code looks like.

\textit{Question 2: What would you like to understand more about the code in order to enjoy the performance more?}

In some ways I would like to understand a little more about the code. It would be nice to have a director's commentary of what's going on behind the scenes, just so I can follow along with the changes that I can hear in the music as they are occurring. But I think I more enjoy just listening to the music, knowing broadly that a livecoder is manipulating code to make the sounds that I hear. I don't often like reading the code for the whole performance, maybe for a few minutes at a time, but then I like to switch off and just focus on what the musician is playing. I more often like to listen to the music and guess what the livecoder has done to make that changes (which is kind of backwards). I wouldn't mind having more understanding of the code on hand, but I probably wouldn't use the details of it during the whole performance every time.
