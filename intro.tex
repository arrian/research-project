%%
%% Template intro.tex
%%

\chapter{Introduction}
\label{cha:intro}

-code is often difficult to quickly understand
-some observers may lack the experience to understand the software or the programming process

-how can we improve source code comprehension?
-how can we aid understanding of the programming process?
-better yet, how can we better communicate the programmers intention?

-techniques such as modelling or code documentation aren’t dynamic or flexible
-don’t allow for close to realtime understanding
-an effective technique is the use of visualisations
-it would be valuable to use visualisations as a means to communicate the programmer’s intention


\section{Summary - remove}

-this thesis will explore code visualisations
-specifically, it will investigate visuals within the combination of the domains of software and music
-will be using Iive coding as a platform and case study for this (will discuss later)
-will develop and test code visualisations on audiences with audiences of varied levels of experience with programming, addressing code comprehension

\section{Background}
\label{sec:basis}



\section{Theoretical Framework}
\label{sec:framework}

%%% Local Variables: 
%%% mode: latex
%%% TeX-master: "thesis"
%%% End: 
