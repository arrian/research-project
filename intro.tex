%%
%% Template intro.tex
%%

\chapter{Introduction}
\label{cha:intro}

-code is often difficult to quickly understand
-some observers may lack the experience to understand the software or the programming process

-how can we improve source code comprehension?
-how can we aid understanding of the programming process?
-better yet, how can we better communicate the programmers intention?

-techniques such as modelling or code documentation aren’t dynamic or flexible
-don’t allow for close to realtime understanding
-an effective technique is the use of visualisations
-it would be valuable to use visualisations as a means to communicate the programmer’s intention


\section{Summary - remove}

-this thesis will explore code visualisations
-specifically, it will investigate visuals within the combination of the domains of software and music
-will be using Iive coding as a platform and case study for this (will discuss later)
-will develop and test code visualisations on audiences with audiences of varied levels of experience with programming, addressing code comprehension

\section{Background}
\label{sec:basis}


Live Coding
-live coding is a platform for bridging these two domain visualisations 
-what is live coding?
-method of programming in front of an audience for artistic or informative purposes
-the live coder displays their screen to an audience, showing their code as they are working on it building a functional program
-makes use of interactive programming environments 
-program running while changes are being made 
-often focusses on improvisation - the programmer often has to think on their feet

-what does live coding achieve?
-gives the audience insight into the programming process - i’ll be taking advantage of this


\section{Theoretical Framework}
\label{sec:framework}

Research Project - Visualisation Taxonomy


Emotion-based Music Visualisation using Photos
-Classifications including: sublime, sad, touching, easy, light, happy, exciting, grand

http://www.aviz.fr/Research/ActivePhysicalVisualizations




TAXONOMY

Both code and music present a wide variety of visualisation techniques. These techniques will be summarised below.

Music Visualisation

Generative Visualisation
-“generates animated imagery based on a piece of music”
-eg. change with loudness and frequency spectrum
-VLC, iTunes etc.
-

Visualisations based on Association/Story Telling
-eg. video art, sampled video with sampled music
-

Domain Specific Visualisation
-could include music creation tools, for example abelton etc.
-graphic representations that have one-to-one mapping
-direct visualisation
-understood by the domain
-sheet music


Code Visualisation

Visualisations based on Intent
-unkown…

Visualisations based around Code Augmentation
-infographics
-annotations (visual code annotations for cyberphysical programming)
-sparklines (Visual Monitoring of Numeric Variables Embedded in Source Code)
-etc

Visuals based on Abstraction
-lego mindstorms
-code flow

Visualisations based around Analysis
-debugging
-tracing

Domain Specific Visualisations
-eg. UML, eclipse, visual studio
-fluid source code views (Fluid Source Code Views for Just In-Time Comprehension)
-class diagrams




-visualisations that respond to music
-visualisations that respond to typing



-Goal: categorising existing visualisations

-Infovis discussed in Rethinking Visualization: A High-Level Taxonomy
-Taxonomy developed within this article consists of {discrete, continuous} vs display attributes (eg. given, constrained, chosen) per the design model



Taxonomy of Code Visualisations

-Discussed in A Principled Taxonomy of Software Visualisation (1993)
-Myers (1986) classifies using level of abstraction vs level of animation. (Visual Programming, Programming by Example… A Taxonomy) Also uses {static, dynamic} vs {code, data}. Minimal discussion of dynamic visualisations; no elaboration.
-


Taxonomy of Music Visualisations

-music is similar to software in a number of ways
	-often has standardised notation
	-expression may diverge from notation
	-visual representation is by default static
	-can be visualised using dynamic methods
	-





Abstract
Semiotic
Domain Specific




Information visualisation vs Scientific visualisation - “infovis when spacial representation is chosen, scivis when spacial representation is given”




Information visualisation



Visualisation Features
-Shape
-Size
-Orientation
-Dimensionality
-Colour


Self-illustrating phenomena



Basis

Rethinking Visualization: A High-Level Taxonomy
-Old method of categorisation: Scientific vs Information (incl. factors such as scientific vs non-scientific, physical vs abstract, spacialisation given vs spacialisation chosen)
-Introduce 'model based' visualisation techniques
-“our main objective is to provide insight into how different research areas relate, not to provide guidelines for visualisation design."
-Terminology - Object of study, Data, Design model, User model (see image in article for relationship)
-Taxonomy developed within this article consists of {discrete, continuous} vs display attributes (eg. given, constrained, chosen) per the design model
-Continuous model visualisation is broken down according to the number if independent and dependent variables and the type of the dependent variables (incl. scale, vector and tensor)
-Discrete model visualisations are broken down into wether the data structure or data values are visualised

-Articles discusses lower level taxonomy including - spacial relationships, numeric trends, patterns, connectivity, filtering


Semiology of Graphics: Diagrams, Networks, Maps
-


%%% Local Variables: 
%%% mode: latex
%%% TeX-master: "thesis"
%%% End: 
