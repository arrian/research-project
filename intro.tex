%%
%% Template intro.tex
%%

\chapter{Introduction}
\label{cha:intro}

-code is often difficult to quickly understand
-some observers may lack the experience to understand the software or the programming process

-how can we improve source code comprehension?
-how can we aid understanding of the programming process?
-better yet, how can we better communicate the programmers intention?

-techniques such as modelling or code documentation aren’t dynamic or flexible
-don’t allow for close to realtime understanding
-an effective technique is the use of visualisations
-it would be valuable to use visualisations as a means to communicate the programmer’s intention


\section{Summary - remove}

-this thesis will explore code visualisations
-specifically, it will investigate visuals within the combination of the domains of software and music
-will be using Iive coding as a platform and case study for this (will discuss later)
-will develop and test code visualisations on audiences with audiences of varied levels of experience with programming, addressing code comprehension

\section{Background}
\label{sec:basis}


Live Coding
-live coding is a platform for bridging these two domain visualisations 
-what is live coding?
-method of programming in front of an audience for artistic or informative purposes
-the live coder displays their screen to an audience, showing their code as they are working on it building a functional program
-makes use of interactive programming environments 
-program running while changes are being made 
-often focusses on improvisation - the programmer often has to think on their feet

-what does live coding achieve?
-gives the audience insight into the programming process - i’ll be taking advantage of this


\section{Theoretical Framework}
\label{sec:framework}

%%% Local Variables: 
%%% mode: latex
%%% TeX-master: "thesis"
%%% End: 
