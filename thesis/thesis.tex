\documentclass[11pt]{book}

\usepackage[palatino]{anuthesis}
\usepackage{graphicx}
\usepackage{thesis}
\usepackage{makeidx}
\usepackage{acmnew-xref}%edited `Bibliography' title to `References'

%% Custom Arrian Stuff
\usepackage[titletoc]{appendix}
\usepackage{acronym}
\usepackage{pdfpages}
\usepackage{afterpage}
\usepackage{amsmath}
\usepackage{subcaption}

% for text numbered as equations
\newcounter{question}[chapter]
\renewcommand{\thequestion}{\thechapter.\arabic{question}}
\makeatletter
\newcommand{\qlab}[1]{\leavevmode\hfill\refstepcounter{question}\label{#1}\textup{\tagform@{Question \thequestion}}}
\makeatother

% for guidelines
\newcounter{guideline}[chapter]
\renewcommand{\theguideline}{\thechapter.\arabic{guideline}}
\makeatletter
\newcommand{\glab}[1]{\leavevmode\hfill\refstepcounter{guideline}\label{#1}\textup{\tagform@{Guideline \theguideline}}}
\makeatother

% for `inspirational' quotes - from http://tex.stackexchange.com/questions/53377/inspirational-quote-at-start-of-chapter
\makeatletter
% \renewcommand{\@chapapp}{}% Not necessary...
\newenvironment{chapquote}[2][2em]
  {\setlength{\@tempdima}{#1}%
   \def\chapquote@author{#2}%
   \parshape 1 \@tempdima \dimexpr\textwidth-2\@tempdima\relax%
   \itshape}
  {\par\normalfont\hfill--\ \chapquote@author\hspace*{\@tempdima}\par\bigskip}
\makeatother


\usepackage{color}
\newcommand{\citetemp}[1]{{\color{red}[#1]}}
% usage example: \citetemp{ref}

\newcommand{\more}{{\color{red} [more here]}}
\newcommand{\fix}{{\color{red} [fix this]}}


% Acronyms
\acrodef{IDE}[IDE]{integrated development environment}
\acrodef{OSC}[OSC]{Open Sound Control}
\acrodef{SoW}[SoW]{state of the world}
\acrodef{SoC}[SoC]{state of the code}
\acrodef{SDLC}[SDLC]{software development life cycle}
\acrodef{REPL}[REPL]{read-eval-print-loop}
\acrodef{SLOC}[SLOC]{source lines of code}

% --------------------------------------------------
\title{Art and Understanding through Code Visualisation}
\author{Arrian Purcell}
\date{\today}

\renewcommand{\thepage}{\roman{page}}

\makeindex
\begin{document}


\pagestyle{empty}
\thispagestyle{empty}
\input 1-titlepage

\input 2-frontmatter

% \cleardoublepage
% \pagestyle{empty}
% \input dedication

\cleardoublepage
\pagestyle{empty}
\input 3-quote

\cleardoublepage
\pagestyle{empty}
\input 4-acknowledgement

\cleardoublepage
\pagestyle{headings}
\input 5-abstract

\cleardoublepage
\pagestyle{headings}
\markboth{Contents}{Contents}

{\tableofcontents \let\cleardoublepage\clearpage \listoffigures}

% \tableofcontents
% \listoffigures
% \listoftables

\mainmatter

\input 6-intro
\input 7-literature-review
\input 8-field-study
\input 9-visualisation-design
\input 10-user-study
\input 11-visualisation-refinement
\input 12-follow-up-user-study
\input 13-summary
\input 14-conclusion
\input 15-appendix

\backmatter

\cleardoublepage % leave page between appendices and references
\bibliographystyle{acmnew-xref}% \bibliographystyle{ieeetr}
\bibliography{thesis}

\printindex

\end{document}


% -----------------------
% Questions to ask (thesis post-mortem):

% What was your thesis about?
% What did you find out?



