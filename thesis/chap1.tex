%!TEX root = thesis.tex

\chapter{Literature Review}

Software visualisation is building momentum within the space of live coding. This section seeks to identify the reason for this momentum and identify the purpose and potential for visualisations within this field.

\section{Software}

Understanding changing software is one of the most important goals within the space of software engineering practice~\cite{Tao2012}. It is the nature of software to change~\cite{Purushothaman2005} and there is a need for not only the programmer to understand the software but also a knowledge transfer to take place between those creating the software and those observing the software as it changes~\citetemp{ref}.

Programming languages are the formal languages of software. These programming languages are typically represented by source code, generally in a plain text format~\cite{Badros2000}. \ldots Plain text format is limited requiring an interpretation step (parsing and compilation) to acheive a functional program~\cite{Badros2000}.

The concept of alternative source code representations is not new. Alternative source code representations include static diagrams and visual languages. These vary greatly in the relation to the source code with some representations presenting a highly simplified code representation when compared to the complete source code representation.

Although it is the nature of software to change, static diagrams have traditionally been used to represent software systems. These diagrams typically show the structure (class diagrams) or function (state diagrams) of a software system at a specific moment in time.\\
-usefulness of these diagrams lies in ability to represent fundamental structures within the program\\
-usefulness declines as they become more precise and less abstract -- must be generated continuously to keep up with the changing software.

Visual languages are targetted at the programmer, providing an alternative method of interaction with the software development process. However, visual languages often limit the programmer, providing only a limited subset of a full-featured text-based language.

\section{Visualisation}

Visualisation 
What is the purpose of visualisations?
The reason for visualising code must first be identified in order to successfully evaluate visualisation techniques.\\
-why must we identify the reason?\\
-how does the reason for visualising code help to evaluate visualisation techniques?\\
-why do we need to evaluate visualisation techniques?

% some foundational softvis references
Software visualisation is the process of representing the characteristics of computer programs visually~\cite{Stasko1992}. The advantage of providing a visual representation over the more traditional text-based representation is that the text-based approach does not take full advantage of the human visual information processing capability~\cite{Myers1989}.

Initial efforts to classify software visualisations identified the two axes: whether the visualisation illustrated the code, the data or the algorithm and whether the visualisation was static or dynamic~\cite{Myers1989}. Following taxonomies characterised software visualisations according to the aspect of the program, the abstractness of the visualisation, the animation and the automation of the visualisation~\cite{Stasko1992}.


Traditional software visualisation falls into a variety of categories. Each visualisation category corresponds to a reason for visualising the code.\\
-what are the categories?\\
-what reason for visualising the code do the categories correspond to?

The (abstract/graphical/animated) approach provides the basis for effective visualisations. Effective visulisations...\\
-what do effective visualisations do?\\
-what do effective visualisations achieve?\\
Effective software visualisations contribute to making software easier to understand, reflecting the software's history through the lifecycle, facilitating the transfer of knowledge from the programmer to the observer, making important structures visible and managing software complexity~\cite{Baecker1995}.

Visualisations have the capacity to present information more effectively than traditional programming languages. Nevertheless, software visualisations still require significant development to benefit in the understanding of the complexity of software ~\cite{Baecker1995}.

\section{Live Coding}

Live coding can be broadly defined as writing a program while it runs~\cite{Ward2004}. More specifically, live coding is identified as the artistic process of musical and visual expression through programming~\cite{Collins2003}.

Live coding is in a unique position to combine both source code and traditional visualisation techniques~\cite{McLean2010a}. This is due to the approach of live coding to the process of developing software involving effective sensory communication, a goal of exposing audiences to the coding process, fast software iteration and refinement, and an ability to retain an audience's attention. These approaches involved are outlined below.\\
-how can it combine source code and traditional visualisation techniques?

Live coding is built around communicating visually and audibly the software process. It provides a space in which effective means of developing visualisations is provided. \\
-what kind of space is live coding?\\
-what are the effective means of developing visualisations?

Live coding has a history exposing audiences to code.\\
-what is the history of live coding?\\
-how does live coding expose the audience to code?\\
-a human is visible making creative descisions demonstrating a link between human actions and artistic output \cite{Mclean}.

Visualisations can be quickly refined within live coding.\\
-how can visualisations be quickly refined?

Live coding can retain audience attention.\\
-how does live coding maintain attention?\\
-can retain attention through the output (music), the process (programming) or the field (technology).

Live coding is a good application space for visualisations.\\
-why is it a good application space?

Live coding is a good case study for visualisations.\\
-why is it a good case study?\\
-why is it good for visualisations?

Why live code?\\
-what does liveness mean? \cite{Auslander}\\
-what does it contribute?\\
-relation to improvisation?\\
-computational creativity. \cite{Mclean}

During live coding performances, audiences can feel excluded due to an inability to understand the projected source code~\cite{McLean2010a}.

\section{Understanding and Enjoyment}

There is currently a search for visualisations that increase understanding and enjoyment~\cite{McLean2010a}. Here understanding refers to the ability of an observer or audience to \\
-why does he identify understanding?\\
-what is understanding?\\
-why does he identify enjoyment?\\
-what is enjoyment?\\
-how can these two factors contribute to effective visualisations?\\
-the concept assignment problem. \cite{Biggerstaff1994} (this paper may also be useful for explaining what understanding is in the context of software)

Understanding involves a number of factors.\\
-what factors?\\
-how do these factors relate to visualisations?

-understanding software is harder while it is changing. \cite{Eisenbarth2003}\\
-understanding software feature implementation is one of the fundamental challenges to comprehending the code.\cite{Eisenbarth2003}

Similarly, enjoyment involves a number of factors.\\
-what factors?\\
-how do these factors relate to understanding?\\
-how do these factors relate to visualistions?

Duality of understanding and enjoyment...\\
-enjoyment is not completely separable from understanding. (not dual?)\\
-why can we not separate the two?\\
-what is the duality?\\
-how does this relate to visualisations?


\section{Didacticism and Aestheticism}

The duality of the concepts of understanding versus enjoyment are commonly associated with the duality of the concepts of didacticism vs aestheticism.\\
-what is the duality of didacticism vs aestheticism?

Didacticism...\\
-what is didacticism?\\
-how does it relate to visualisations?\\
-how does it relate to education?

Aestheticism...\\
-what is aestheticism?\\
-how does it relate to visualisations?\\
-how does it relate to art?\\
-how does it relate to music?\\
-how does it relate to live performance?\\
-how does it relate to live coding? \cite{Bell}\\
Live coding has much potential to improve the aesthetics of 

Didacticism vs aestheticism...\\
-can we combine the two?\\
-can we completely separate the two?

