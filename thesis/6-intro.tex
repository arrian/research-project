%!TEX root = thesis.tex

\chapter{Introduction}
\label{chap:introduction}

\begin{chapquote}{Frederick Brooks, \textit{The Mythical Man-Month}, p.185}
``In spite of progress in restricting and simplifying the structures of software, they remain inherently unvisualizable, thus depriving the mind of some of its most powerful conceptual tools. This lack not only impedes the process of design within one mind, it severely hinders communication among minds.''
\end{chapquote}


A fundamental challenge of modern software is of utility. Questions regarding how software interacts with the world and how those that use the software benefit from it drive much of modern software engineering research. There is no doubt software is becoming more useful. However, software is also becoming more complex and the methods of interacting with software are diversifying.

Live coding is one area in which the utility of software has not been thoroughly examined. Live coding, the artistic process of programming for an audience, asks the performer (the live coder) to show the audience their screens. This results in many live coding performances projecting raw source code for the audience. Immediately, this raises two questions: ``why show the screen at all?'' for audiences that have no programming experience; and ``what can be gained by showing the source code?''. Answering these questions would not only have immediate benefit in understanding how to enhance the audience's experience during live coding performances but could also have the wider benefit of communicating the programming process.

Visualisations have been suggested as a method of assisting the audience in the appreciation and comprehension of live coding performances. Historically in software, visualisations have attempted to provide a higher level of understanding of the structure of programs but have rarely been effective in communicating the fundamental processes of software development. \textit{The Mythical Man-Month} goes so far as to suggest that ``software is invisible and unvisualizable''\cite{Brooks1995} sparking debate as to the effectiveness of current software visualisation techniques and whether software can be visualised at all.

There has been a push to develop effective software visualisations as the audience of software becomes more diverse, multidisciplinary teams within the industry become the standard, and multimedia and the arts become more focussed on effective software development. Live coding presents a space in which a diverse audience and dynamic software is common and in which visualisations could provide significant benefit to the experience of observers. Live coding is a means of combining the artistic goal of enjoyment and the educational aspects associated with programming with visualisations to effectively communicate software.

Recently, the development of live coding has presented a unique application space for visualisations and the potential for effective communication directly with audiences. These new developments could challenge the long-held belief that ``software is invisible and unvisualizable''.

This thesis investigates the proposition that ``code visualisations improve the experience of observers'' in the setting of live coding. More specifically, this thesis investigates the question: ``can the application of visualisation techniques to live coding enhance audience experience by increasing understanding and enjoyment?''. These questions were examined through a process of prototype development, user study evaluation and refinement.

%!TEX root = ../thesis.tex

\begin{figure}
\centering
\begin{subfigure}{.5\textwidth}
\centering
  \includegraphics[width=.95\linewidth]{../images/code-visualisations/class-diagram.pdf}
  \caption{Class diagram}
  \label{fig:class-diagram}
\end{subfigure}%
\begin{subfigure}{.5\textwidth}
\centering
  \includegraphics[width=.8\linewidth]{../images/code-visualisations/flow-chart.pdf}
  \caption{Flow chart}
  \label{fig:flow-chart}
\end{subfigure}\\
\vspace{5mm}

\begin{subfigure}{.5\textwidth}
\centering
  \includegraphics[width=.8\linewidth]{../images/code-visualisations/sequence-diagram.pdf}
  \caption{Sequence diagram}
  \label{fig:sequence-diagram}
\end{subfigure}%
\begin{subfigure}{.5\textwidth}
\centering
  \includegraphics[width=0.8\linewidth]{../images/code-visualisations/bundle-graph.png}
  \caption{Edge bundled graph \protect\cite{Holten2006}}
  \label{fig:bundle-graph}
\end{subfigure}

\caption[Existing software diagramming and graphing techniques]{Existing software diagramming and graphing techniques.}
\label{fig:code-diagrams}
\end{figure}

\section{Background}

For most of its history, program source code has been displayed as simple text. This is due to the expressiveness of the text format and despite the inefficiencies of representing algorithms and abstractions in natural language.

Recently, due to ever increasing programming language complexity, increasing screen fidelity and increasing computational power, code annotations and syntax highlighting have become commonplace. Nevertheless, these visual enhancements rarely provide information beyond the basic grammar of the language they are intended to augment. The limitations of this approach are becoming ever more apparent as programming languages and interactive programming environments move towards the need for real-time comprehension and a need to understand the source code within the context of a running program.

Historically, source code diagrams have attempted to display the high level structure of source code. For example, class diagrams (see Figure~\ref{fig:class-diagram}) are commonly used to document the structure of a program whereas flowcharts (see Figure~\ref{fig:flow-chart}) have been used extensively to represent simple software behaviours. Sequence diagrams (see Figure~\ref{fig:sequence-diagram}) have allowed the visualisation of interacting objects and actors in a software system, putting the focus on the users of the software system rather than the structure of the program. Edge bundling of software call hierarchies~\cite{Holten2006,Zhou2013} (see Figure~\ref{fig:bundle-graph}) has allowed the visualisation of the lifecycle of a program, showing which software components were called at what time and the volume of calls made.

\subsection{Hot Swapping Source Code}

Various modern software environments now support the concept of hot swapping source code. Within software environments, hot swapping refers to the process of modifying program source code at runtime to achieve desired behaviour without interrupting or restarting the system. Examples of languages and platforms supporting this workflow include Java with HotSwap or JRebel~\cite{ZeroTurnaround2014} and Python, in addition to a variety of other custom platforms and modifications (e.g.~\cite{Thomas2011}). This workflow has also inspired the programming practice of ``live coding''. 

Live coding further develops the concept of hot swapping source code, applying it to live musical and visual performance. In live coding a programmer performs for an audience. The audience observes the programmer as they modify an active program to produce music and visuals. Some of the most important and influential live coding environments over the last decade have included SuperCollider~\cite{McCartney}, ChucK~\cite{Wang2008} and Extempore~\cite{Sorensen}. These environments have a growing following as the field of live coding matures.

\subsection{Live Coding}

\begin{figure}
\centering
\includegraphics[width=1.0\textwidth]{../images/code/live-coding-screen.png}
\caption[A typical live coder projection]{A typical live coder projection includes the raw source code of the musical performance such as the screen pictured above. Audiences observe the changes to the source code on a projector screen while listening to the musical output. In the source code pictured here, the live coder is making changes to the musical ``bassline'' during a performance.}
\label{fig:live-coding-screen}
\end{figure}

Live coding can be broadly defined as writing a program while it runs~\cite{Ward2004}. More specifically, live coding is identified as the artistic process of musical and visual expression through programming~\cite{Collins2003}. Figure~\ref{fig:live-coding-screen} shows the standard screen projected for a live coding audience. The audience views the raw source code and the programming process as the programmer makes changes to the source code.

Live coding emphasises the concept of \emph{liveness}, a concept covered comprehensively within the literature~\cite{Auslander,Masura2007}. Liveness is foundational to the performance aspect of live coding in which the mantra is ``show us your screens''~\cite{Toplap}, asserting that all live coders should display their raw program source code to the audience.

A unique opportunity is provided through live coding to combine both source code and software visualisation techniques~\cite{McLean2010a}. This is due to the approach of live coding involving effective sensory communication, a goal of transparency of the coding process, and direct manipulation and refinement of the running program.

Firstly, live coding is built around communicating visually and audibly. This extends beyond the physical process of programming. Human input is visible in the creative process, demonstrating a link between physical actions and artistic output \cite{Mclean}.

Secondly, live coding has a history of exposing audiences to source code with the goal of ensuring the transparency of the coding process~\cite{Collins2011,McLean2010a}. It provides a space in which there is a direct mapping from the human interaction with the source code to the musical or visual output~\cite{Mclean}. This relationship allows for visuals to map the interaction with the output.

Lastly, live coding allows for direct manipulation and refinement of the running program~\cite{Swift2013}. This provides the capacity for uninterrupted visualisation of the dynamic aspects of the program combined with the static nature of the source code. Similarly, manipulation and refinement of the running program allows for manipulation of the software visualisation~\cite{McLean2010a}.

\section{Related Work}

\begin{figure}
\centering
\begin{subfigure}{.5\textwidth}
  \includegraphics[width=.95\linewidth]{../images/code-visualisations/syntax-highlighting.png}
  \caption{Syntax Highlighting}
  \label{fig:syntax-highlighting}
\end{subfigure}%
\begin{subfigure}{.5\textwidth}
  \includegraphics[width=.95\linewidth]{../images/code-visualisations/gource.png}
  \caption{Gource}
  \label{fig:gource}
\end{subfigure}\\
\begin{subfigure}{.5\textwidth}
  \includegraphics[width=.95\linewidth]{../images/code-visualisations/code-swarm.png}
  \caption{Code Swarm}
  \label{fig:code-swarm}
\end{subfigure}%
\begin{subfigure}{.5\textwidth}
  \includegraphics[width=.95\linewidth]{../images/code-visualisations/scheme-bricks.png}
  \caption{Scheme Bricks}
  \label{fig:scheme-bricks}
\end{subfigure}

\caption[Existing software visualisations techniques]{Existing software visualisations techniques.}
\label{fig:code-visualisations}
\end{figure}  

As noted above, a number of existing software visualisation techniques exist. Some of the most common software visualisations include static diagrams (see Figure~\ref{fig:code-diagrams}) and syntax highlighting (see Figure~\ref{fig:syntax-highlighting}). These techniques generally model the structure of the program to allow for more effective navigation around structures and take advantage of visual memory.

While diagrams and syntax highlighting provide information regarding the structure of source code, the fundamental nature of software is to change~\cite{Brooks1995}. Understanding these changes requires visualising the software development process. Techniques to visualise the changes occurring during the software development process include Gource~\cite{Caudwell2010} (Figure~\ref{fig:gource}) and Code Swarm~\cite{Ogawa2012} (Figure~\ref{fig:code-swarm}). These visualisation techniques show historic source code changes, based on source code repository data.

Showing historic source code changes allows the process of programming to be understood. However, this information is not immediately useful to programmers or observers. Visualisation of active software has shown potential in providing useful and interesting information, allowing the programmer to take actions based on the visualisations and allowing the observers to make relevant judgement of what actions the programmer is taking. Visualisation techniques such as inline annotations~\cite{Swift2013,Beck2013} provide direct feedback useful to both the programmer and observers, and technologies such as Light Table~\cite{Kodowa2014} with an inline interactive terminal and immediate visual feedback after hot swapping source code have further identified how useful visual feedback can be.

A number of attempts have been made in visualising active software within live coding. Examples of existing software visualisations within live coding include Scheme Bricks (Figure~\ref{fig:scheme-bricks}), Daisy Chain and Betablocker~\cite{McLean2010a}. These visualisation techniques suggest potential within live coding for effective process-driven visuals to enhance the observer's experience. However, no systematic studies of process-driven visualisations have been conducted within live coding.

\section{Structure}

This thesis discusses software visualisations which have been applied to live coding through a process of design iteration and evaluation in collaboration with a live coding artist. Three studies were administered, including one field study and two laboratory studies as a means to determine if understanding and enjoyment could be influenced by the introduction of visualisations to live coding.

Chapter~\ref{chap:literature-review} of this thesis summarises the literature including the basis for and the direction of this thesis. The current state of software visualisations and the state of the application of visualisations to live coding are examined with particular reference to the goals and future directions of the field.

Chapter~\ref{chap:exploratory-field-study} discusses the initial exploratory field study conducted to investigate existing perception and understanding of the live coding process. The field study explored the current state of live coding practice, examining the process of live coding, the effect on the audiences of live coding and the expectations of the audience.

Chapter~\ref{chap:visualisation-design} discusses the first iteration of the visualisation prototype developed following the results of the exploratory field study. The process of development is described with particular focus on software design within the space of an artistic process. This process involved collaboration with a live coding artist to develop software visualisations appropriate for the space of live coding.

Chapter~\ref{chap:user-study} summarises the initial user study conducted with the visualisation prototype. This study compared visualisation features in reference to enhancing the experience of the live coding audience. The study used surveys and video-cued recall techniques to examine live coding in the context of an artistic practice.

Chapter~\ref{chap:visualisation-refinement} discusses refinement of the visualisation prototype driven by the results of the initial user study. The initial user study identified a number of areas of improvement within the first iteration of the visualisation prototype and provided direction for following prototypes. The visualisation refinement took these concepts and applied them to the first iteration of the visualisation prototype.

Chapter~\ref{chap:follow-up-user-study} discusses the follow-up user study conducted to analyse the refined visualisations. A similar approach to the initial user study was taken, examining live coding audiences within the context of a live coding performance. However, in this case the examination compared the refined visualisation prototype with presenting just the plain source code within the live coding performance.

Chapter~\ref{chap:conclusion} summarises the results of the user studies, contributions, limitations and future work. Results of the field study and the two user studies are compared and the findings are discussed. Following this, the contributions made to the field of software visualisation and the visualisation of live coding are listed. Limitations with the contributions and limitations of the studies are then discussed. Future work in the field of software visualisation and the visualisation of live code is discussed with particular reference to developments of the methodology, improvements to the user studies and application spaces of the visualisation techniques discussed. A reflection on the process of combining a scientific approach with the arts is presented, followed by some closing words to conclude the thesis.


