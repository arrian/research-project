%!TEX root = thesis.tex

\chapter*{Abstract}
\label{chap:abstract}
\addcontentsline{toc}{chapter}{Abstract}

The visualisation of source code has been examined as a method of enhancing audience experience within ``live coding''. Following an exploratory field study examining the experience of the audience observing source code projections, two process-driven source code visualisation techniques were incorporated into a live coding system. These visualisations were evaluated within a user study during a series of live coding performances, examining the self-reported experiential dimensions of \emph{understanding} and \emph{enjoyment}. Through a process of refinement, a follow-up user study was conducted on a second iteration of the visualisation techniques comparing visualisations with no visualisations to determine if the visualisations developed had enhanced the audience's experience.

The exploratory field study examined the validity of the dimensions of \emph{understanding} and \emph{enjoyment} for evaluating visualisations within live coding and indicated the potential for visualisations to enhance these dimensions through the live coding performance. Results of the intial user study identified the ability to manipulate the dimensions of understanding and enjoyment through the \emph{beginning}, \emph{middle} and \emph{end} of the live coding performance. Finally, the follow-up user study determined the effectiveness of the visualisations over no visualisations within live coding. Minor increases in both understanding and enjoyment were identified for these visualisations, with an identified increase in understanding for those initially with lower understanding and an increase to enjoyment throughout the middle stages of the live coding performance. Results of the three studies are indicative of the potential for visualisations within live coding in enhancing the experience of the audience. 

