%!TEX root = thesis.tex

\chapter{Follow-Up User Study}
\label{chap:follow-up-user-study}

A follow-up user study was conducted to evaluate the set of refined visualisations. In addition to evaluating this iteration of the visualisation prototype, the limitations identified in the first user study were also to be addressed. This included providing a valid baseline for the study (the \emph{no visualisation} condition) and reducing the visualisations to more meaningful representations.

% The previous visualisation prototype and user study also lacked a clear baseline.  with an analysis and comparison to a live coded performance without -previous study did not have a baseline - this was to be remedied

It was hypothesised that the visualisation prototype would result in higher understanding at the end of the performance and that enjoyment would remain steady throughout both performances.

\section{Method}

Two independent audiences ($N=14+11=25$) were recruited through on campus advertisement (see Appendix~\ref{appendix:follow-up-user-study-advertisement}). Each group was exposed to two live coding musical performances. One of the performances displayed only the source code of the performance while the other displayed the source code with the refined visualisation prototype as an underlay (see Chapter~\ref{chap:visualisation-refinement}).

The first group was subjected to the \emph{visualisation} condition, followed by the \emph{no visualisation} condition. The conditions were swapped for the second group, with the audience exposed to the \emph{no visualisation} condition first followed by the \emph{visualisation} condition.

\section{Participants}

Of the 25 total participants over the two performances $12\%$ were female. $64\%$ of the audience had not been to a live coding performance before and $80\%$ had not attended the previous live coding user study. The background of the audience included $56\%$ that listened to music regularly, $28\%$ that played an instrument and $72\%$ who programmed for their hobby, job or study.

\section{Results}

The audience-reported enjoyment and understanding responses from the survey were evaluated for the two conditions as described below.

\subsection{Enjoyment}

$60\%$ of the audience stated that projection of the code helped their enjoyment of the performance directly while $16\%$ of the audience stated that the visualisations helped their enjoyment directly.

Levels of enjoyment throughout the performance were surveyed for the \emph{beginning}, \emph{middle} and the \emph{end} phases of the performance. A comparison of the enjoyment between the no visualisation condition and the visualisation condition is available in Figure~\ref{fig:no-visualisations-enjoyment} and Figure~\ref{fig:visualisations-enjoyment}.

Final levels of enjoyment between the two conditions differed only slightly. With no visualisation, $20\%$ stated that they had low enjoyment, $40\%$ stated that they had medium enjoyment and $40\%$ stated that they had high enjoyment. With visualisations, $20\%$ stated low enjoyment, $36\%$ stated medium enjoyment and $44\%$ stated high enjoyment.

\afterpage{
\begin{figure}
  \centering
  \includegraphics[width=\columnwidth]{../study-3/results/enjoyment-with-no-visualisation-study-3}
  \caption{Audience reported enjoyment during the beginning, middle and end of the performance with \textbf{no} visualisations. Line width at each stage indicates proportion of the audience reporting high, medium or low enjoyment, and line colour is determined by the enjoyment level at the \emph{beginning} of the performance.}
  \label{fig:no-visualisations-enjoyment}
\end{figure}

\begin{figure}
  \centering
  \includegraphics[width=\columnwidth]{../study-3/results/enjoyment-with-visualisation-study-3}
  \caption{Audience-reported enjoyment level during the beginning, middle and end of the performance with visualisations.}
  \label{fig:visualisations-enjoyment}
\end{figure}
\clearpage}

\subsection{Understanding}

$76\%$ of the audience stated that projection of the code helped their understanding of the performance directly while $32\%$ of the audience stated that the visualisations helped their understanding directly.

Levels of understanding throughout the performance were surveyed for the \emph{beginning}, \emph{middle} and the \emph{end} phases of the performance. A comparison of the understanding between the no visualisation condition and the visualisation condition is available in Figure~\ref{fig:no-visualisations-understanding} and Figure~\ref{fig:visualisations-understanding}.

Levels of understanding also differed slightly between the two conditions. With no visualisation, $32\%$ stated that they had low understanding, $48\%$ stated that they had medium understanding and $20\%$ stated that they had high understanding. With visualisations, $28\%$ stated low understanding, $44\%$ stated medium understanding and $28\%$ stated high understanding.

\afterpage{
\begin{figure}
  \centering
  \includegraphics[width=\columnwidth]{../study-3/results/understanding-with-no-visualisation-study-3}
  \caption{Audience reported understanding during the beginning, middle and end of the performance with \textbf{no} visualisations.}
  \label{fig:no-visualisations-understanding}
\end{figure}

\begin{figure}
  \centering
  \includegraphics[width=\columnwidth]{../study-3/results/understanding-with-visualisation-study-3}
  \caption{Audience reported understanding during the beginning, middle and end of the performance with visualisations.}
  \label{fig:visualisations-understanding}
\end{figure}

\clearpage}

After each performance, two additional questions were asked. The additional questions were ``what do you think the performer was doing in the very early statges of the performance?'' and ``what do you think the performer was doing in the very last stages of the performance?''. Results from these questions were evaluated against a criteria for understanding level. {\color{red} discuss criteria here}

\subsection{Liveness}

After exposing the audience to both conditions, the audience was asked if the projected visualsations helped to communicate the feeling that the performances were live. Results of this question indicate that...

\more

\section{Discussion}

\more



\subsection{Validity}

A number of factors were identified that could influence the validity of the user study results.

Due to a technical oversight during the first performance, the visualisation underlay was difficult to distinguish. 
% This was this was intentionally left the same for the second performance group.

A large proportion of the audience stated that they had coding experience. An audience experienced in coding may be less interested in viewing visualisations and have a preference for viewing code directly.

-no significance in the results

Again, musical style... instruments... differences between two performances etc

\more

\subsection{Limitations}

-this set of visualisations again suffered from being obscured by the code displayed on the screen, however, compared to the second study the visualisations were logically placed in relation to the source code... from top to bottom, in function order.

\more

\section{Summary}

\more








