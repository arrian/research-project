%!TEX root = thesis.tex

\chapter{Summary}
\label{chap:summary}

A number of studies have been conducted to determine if the application of visualisation techniques to live coding can enhance audience experience. Results of the initial exploratory field study examined the live coding space without the application of visualisation techniques. Potential for increasing audience experience was identified and the dimensions of understanding and enjoyment were identified as effective in evaluating the audiences of live coding. Furthermore, this study and the related follow-up email interviews identified that audiences would respond to visualisations that identified the high level aspects of the source code and running program.

Results of the user study identified useful features of two sets of visualisations applied to live coding, an aesthetic approach and a didactic approach. The results of the survey and the results of the follow-up interview conducted with the live coder identified that the mental models of the audience and the live coder differed greatly and that effective visualisation techniques should look to identify and reduce the causes of the differences.

Results of the follow-up user study demonstrated that visualisations targeting  both audience enjoyment and understanding provided enhancements to the audience experience. Following analysis and comparison of the user study and the follow-up user study suggest that visualisations targeting enjoyment may result in the most positive audience response.

\section{Contributions}

Throughout this investigation of the application of software visualisations to live coding, a number of contributions to the field have been identified. This section outlines these contributions.

Principally, a method of software visualisation has been identified and evaluated. Up to now there has been no attempt to evaluate visualsations within the space of live coding. The method of process-driven visualisation identified has included a combination of static and dynamic code analysis and presentation to audiences during live coding performances. This visualisation approach was built on design guidelines identified with the literature. This has allowed for the application of a novel approach to live coding with a solid theoretical basis.

Through the development of the software visualisation prototypes, a strategy for developing visualisations using a software engineering approach has been identified. This approach has involved collaboration with a live coding artist with the application of a software engieering design, iteration and validation approach through a variety of field and user studies.

The development of the software visualisation prototype and the application to the live coding space have allowed the development of a method of evaluating software visualisations within a live performance space. The more general implications of this method of visualisation evaluation may be useful in the evaluation of software developed for large audiences and audiences observing an individual developing software.

A conference short paper discussing the prototype and evaluation methodology has been accepted to the OzCHI conference entitled ``Visualising a Live Coding Arts Process''. Anonymous references indicated that ``the questions raised are worthwhile and interesting'', ``the paper describes interesting work'' and that it is ``an interesting paper, and certainly a novel contribution''. The acceptance of this paper validates the contributions discussed as relevant to the current state of research within live coding and also the wider field of human-computer interaction.

\section{Limitations}

Associated with these contributions are some limitations within the visualisation design, development and evaluation methodology.

Limitations within the evaluation methodology identified result from a wide variety of sources throughout the field and user studies. The survey focussed heavily on self-reported enjoyment and understanding, and the audience's perception of the stages of the performance including the beginning, middle and end. Although the validity of these measures was identified early within the project, there still appears to be some variance in self-reporting.

All studies conducted were musical in nature and this musical nature presented challenges during survey design. Enjoyment of the performance appeared to centre around the audience's perception of the musical style and their own musical preference. Survey data was collected on audience musical experience, however, musical preference presents an additional challenge in comparing user study participants.

The maturity of the musical nature of live coding also directly impacts visual expressiveness. Music within live coding is far more mature and developed that visualisations and as indicated through these studies, visualisations still have much potential to develop further. It may be that because of the maturity of the musical aspects of live coding, the music influenced perception far more than the visualisations. This was shown in the audience's written survey responses. Responses indicated that issues with the music were more prominent with larger percentages discussing musical limitations over visual limitations.

The exploration of understanding was based around a language with a functional syntax. Few audience members within the initial user study indicated that they had experience with the Lisp family of programming languages, a family commonly considered functional in nature, and it may be that there are differences between this family of languages and more common imperitive or object oriented styles of programming. This presents limitations in the generalisation of these visualisation techniques to other programming languages.

Finally, as hinted, there may be limitations in the generalisation of the evaluation methodology outside the field of live coding. These studies have explored concepts inherent to the application of science to the arts. While the techniques of scientifically evaluating the arts, application of these techniques to more traditional software developments needs further investigation.

\section{Future Work}

The process of implementation and evaluation of visualisations conducted are the first steps towards what is possible within live coding. The process of applying visualisations to live code has indentified areas of future work and areas that would benefit from future systematic evaluation or development. This section identifies these areas. 

\subsection{Evaluation Methodology}

Future work will develop the methodolgy identified. The validity of measuring visualisations based on two dimensions of enjoyment and understanding, and three phases of the live coding performance have been evaluated. Future work will develop these concepts, further articulating visualisation enhancements based on the levels of understanding and enjoyment observed through the performance. 

\subsection{User Studies}

There is still much work to be done in identifying what audiences really find interesting and worthwhile. This study has identified the need for taking an aesthetic approach in developing visualisations, in order to enhance the dimensions of enjoyment and understanding.

Further studies could identify useful aesthetic elements of the software visualisations presented. It may be possible to identify a taxonomy of aesthetic visual elements that enhance the audience experience in future studies. Similarly, further studies could identify elements of the didactic visualisations that contributed to decreased understanding and low enjoyment.

Further development of a combination of the aesthetic and didactic approaches, hinted in the follow-up user study, could develop visualisation techniques that consistently improve audience experience across both enjoyment and understanding throughout live coding performances.

Further development of the visualisation techniques and evaluation methodology could allow for more comprehensive user studies. Solving the technical challenges encountered during this set of studies and increasing differentiation between visual features are some obvious first steps in this direction. Similarly, improving upon the process based on the feedback of the audience could be seen as some immediate steps to improve audience experience within live coding performances. For example, suggestions were made to use two projectors during the live coding performance, with one projected the live coder's source code and one projecting a high level visualisation overview of the programming process and programming state.

\subsection{Application Spaces}

The techniques and methodology described here present a number of opportunities for application within a wide variety of disciplines.

\subsubsection{Live Coding}

Immediately, these visualisations could be adapted to a wider range of live coding performance spaces. The visualisation techniques developed could provide an interactive interface to the live coder for manipulation during the performance. This technique could allow the programmer to provide a more expressive view of their source code, providing relevant and timely details of their code manipulation.

This technique would involve live coded visuals. There is much potential to harness the static and dynamic code structures extracted through theses studies \emph{during} the performance. While the visuals developed through this study were static in their expressive power, live coded visuals could provide more visual interest for the audience and allow for visuals to form dynamically with music. Manipulation of visual elements with a goal of enhancing audience experience throughout the performance would provide the basis for live coding as a true multimedia art form.

\subsubsection{Education}

Future work could examine the more pedagogical aspects identified through the application of visualisations to live coding. There is potential to assist audiences with no or little experience in coding to quickly understand high level concepts within the programming process. Applications within this space would apply visualisations as a means to communicate the programming process more effectively and apply visualisations as a means to more effective programming.

\subsubsection{Software Engineering Practice}

The visualisation methodology described here could be applied to software engineering practice. Software visualisations that interact directly with the programmer are yet to be adopted into mainstream development and there is potential to assist multidisciplinary teams by making the programming process more visible.

\subsubsection{The Arts}

The methodology applied through this study takes a step in the evaluation of an artistic process from a scientific perspective. This is an ongoing challenge and this investigation could be applied to a wider variety of performance arts or artistic processes. How to best apply the scientific process to the artistic process is one of the most challenging questions to come from this research, and further scientific understanding of the arts would be of significant benefit.

\subsection{Reflection}

\more

-discuss more of the approach... difficulties in examining the artistic process from a scientific perspective...

-refer to notes from meeting