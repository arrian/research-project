%!TEX root = thesis.tex

\chapter{Summary}
\label{chap:summary}

A number of studies have been conducted to determine if the application of visualisation techniques to live coding can enhance audience experience. Results of the initial exploratory field study examined the live coding space without the application of visualisation techniques. Potential for increasing audience experience was identified and the dimensions of understanding and enjoyment were identified as effective in evaluating the audiences of live coding. Furthermore, this study and the related follow-up email interviews identified that audiences would respond to visualisations that identified the high level aspects of the source code and running program.

Results of the user study identified useful features of two sets of visualisations applied to live coding, an aesthetic approach and a didactic approach. The results of the survey and the results of the follow-up interview conducted with the live coder identified that the mental models of the audience and the live coder differed greatly and that effective visualisation techniques should look to identify and reduce the causes of the differences.

Results of the follow-up user study demonstrated that visualisations targeting  both audience enjoyment and understanding provided enhancements to the audience experience. Following analysis and comparison of the user study and the follow-up user study suggest that visualisations targeting enjoyment may result in the most positive audience response.

\section{Contributions}

Throughout this investigation of the application of software visualisations to live coding, a number of contributions to the field have been identified. This section outlines these contributions.

Principally, a method of software visualisation has been identified and evaluated. Up to now there has been no attempt to evaluate visualsations within the space of live coding. The method of process-driven visualisation identified has included a combination of static and dynamic code analysis and presentation to audiences during live coding performances. This visualisation approach was built on design guidelines identified with the literature. This has allowed for the application of a novel approach to live coding with a solid theoretical basis.

Through the development of the software visualisation prototypes, a strategy for developing visualisations using a software engineering approach has been identified. This approach has involved collaboration with a live coding artist with the application of a software engieering design, iteration and validation approach through a variety of field and user studies.

The development of the software visualisation prototype and the application to the live coding space have allowed the development of a method of evaluating software visualisations within a live performance space. The more general implications of this method of visualisation evaluation may be useful in the evaluation of software developed for large audiences and audiences observing an individual developing software.

A conference short paper discussing the prototype and evaluation methodology has been accepted to the OzCHI conference entitled ``Visualising a Live Coding Arts Process''. Anonymous references indicated that ``the questions raised are worthwhile and interesting'', ``the paper describes interesting work'' and that it is ``an interesting paper, and certainly a novel contribution''.

\section{Limitations}

Associated with these contributions are some limitations within the visualisation design, development and evaluation methodology.

Limitations within the evaluation methodology identified result from a wide variety of sources throughout the field and user studies. The survey focussed heavily on self-reported enjoyment and understanding, and the audience's perception of the stages of the performance including the beginning, middle and end. Although the validity of these measures was identified early within the project, there still appears to be some variance in self-reporting.

All studies conducted were musical in nature and this musical nature presented challenges during survey design. Enjoyment of the performance appeared to centre around the audience's perception of the musical style and their own musical preference. Survey data was collected on audience musical experience, however, musical preference presents an additional challenge in comparing user study participants.

The musical nature of live coding also directly impacts visual expressiveness. Music within live coding is far more mature and developed that visualisations. It may be that because of the maturity of the musical aspects of live coding, the music influenced perception far more than the visualisations. 

limitations to evaluation methodlogy
-self-reported
-stages of performance
-differences in musical taste
% music within live coding is much more mature than visualisations and the separation of the two was impossible during these studies due to there nature. it is plausible that the music may have influenced audience perception more than the visuals
-difference in visual taste
-experience with programming - many of the studies had a high proportion of programmers
-type of language may affect understanding. Imperitive vs functional etc.

limitations to generalisation
-how wide can the results of this study be generalised outside of the field of live coding.
-large portions of the sample population were programmer's, perhaps those who the visualisations would not be targeted at

limitations to assumptions
-understanding and enjoyment are not necessarily entirely independent. enjoyment may contribute to understanding. understanding may contribute to enjoyment.

limitations to visualisation methodology

\section{Future Work}

The process of implementation and evaluation of visualisations conducted are the first steps towards what is possible within live coding. The process conducted has indentified areas of future work and areas that would benefit from future systematic evaluation or development.

\subsection{Live Coded Visuals}

Live coding currently focusses heavily on musical output. Lessons from the studies carried out here could be applied to live coded visuals. Visuals developed over the course of these studies are, by their nature, very static in the expressive power 

develop live coded visuals - much potential to harness the data that has been extracted. it would be worthwhile to take these ideas, further integrating them into the live coding process or introducing them to other software processes using the methodology identified

Future of visualisations in live coding...

Combination with existing live coded visuals...

\subsection{Evaluation Methodology}

Future work will further develop the methodolgy identified. The validity of measuring visualiusations based on two dimensions of enjoyment and understanding, and three phases of the live coding performance have been evaluated. Future work will develop these concepts, further articulating visualisation enhancements based on these concepts. 

\subsection{User Studies}

There is still much work to be done in identifying what audiences really find interesting and worthwhile. This study has identified the need for taking an aesthetic approach in developing visualisations, in order to enhance the dimensions of enjoyment and understanding. 

Further studies could identify useful aesthetic elements of the software visualisations presented that contributed to the increased enjoyment and understanding over the didactic approach. Similarly, further studies could identify elements of the didactic visualisations that contributed to decreased understanding and low enjoyment.

Further development of a combination of these approaches, hinted at in the follow-up user study, could develop visualisation techniques that consistently improve audience experience across both enjoyment and understanding.

Further development of the visualisation techniques and evaluation methodology could allow for more comprehensive user studies, reducing the technical challenges encountered and increasing differentiation between visual features.

A number of improvements were identified and suggestions presented that could further improve future user studies within this space including using two projectors within the live coding performance, on projecting the live coder's source code and the other presenting a high level visualisation of the programming process and program state.

\subsection{Process-Driven Visualisations}

Software visualisations that interact directly with the programmer are yet to be adopted into mainstream development. Similarly, the visualisation of the live coding process has not yet been adopted into live coding performances.


Visualisation as a means to more effective coding...

\subsection{Application Spaces}

The techniques and methodology described here present a number of opportunities for application.

Immediately, the visualisations could be adapted to a wider range of live coding performance spaces. The visualisation techniques developed could be adapted provide an interactive interface to the live coder for manipulation during the performance. This technique could allow the programmer to provide a more expressive view of their source code, providing relevant and timely details of their code manipulation. 

Application to the arts...

Application to education...
Visualisation as a means to communicate the coding process...

Application to software engineering practice...
More general application to software development


\section{Recommendations}

A number of statements were examined throughout the preceding chapters.
-there is a need for increased audience understanding within live coding
-there is a need for increased audience enjoyment within live coding
-there is a need for increased live coder understanding within live coding
-there is a need for increased live coder enjoyment within live coding
-there is a measurable effect on enjoyment with the visualisations compared to without
-there is a measurable effect on understanding with the visualisations compared to without

-visualisations have a place in live coding
-visualisations have a place in understanding the coding process
-visualisations can communicate the coding process more effectively than without
-visualisations have a place in understanding code structures

-qualitative factors seemed to contribute more to the appreciation of the visualisations. People tended towards suggesting that the visualisations should match the music more than the code?

