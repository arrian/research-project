%!TEX root = thesis.tex

\chapter{Summary}
\label{chap:summary}

% \begin{center}
% \begin{table}
% \begin{tabular}{ l l l l}
% \hline 
%  & Field Study & User Study & Follow-Up User Study\\
% \hline 
% need for increased understanding & & & \\
% \hline 
% need for increased enjoyment & & & \\
% \hline 
% effect on enjoyment & & & \\
% \hline 
% effect on understanding & & & \\
% \hline 
% effect on liveness & & & \\
% \hline 
% positive didactic effect & & & \\
% \hline
% \end{tabular}\\
% \caption{Summary of findings over the three studies for the research questions investigated.}
% \label{table:findings}
% \end{table}
% \end{center}

\section{Contributions}

Updated software visualisation taxonomy built on the existing software visualisation frameworks...

Method of evaluating software visualisation...

Method of developing process driven software visualisation for live coding...

Process-driven software visualisation showing software as it is being developed (live, as in live-coding)...

Evaluation of a process-driven software visualisation...

Contribution to live coding process and outcome... A multi-disciplinary approach, implementation and evaluation of a software application.

Discuss recommendations...

``...methodological issues have to be studied further. This concerns questions like how to design visualizations and how to measure and evaluate the effectiveness of various solutions.''~\cite{VanWijk2005} A methodology for visualisation design has been developed.

\section{Limitations}

-type of language may affect understanding. Imperitive vs functional etc.

-how wide can the results of this study be generalised outside of the field of live coding.

\section{Future Work}

Aesthetic elements of the software...

Didactic elements of the software...

Future of visualisations in live coding...

Application to software engineering practice...

Application to the arts...

