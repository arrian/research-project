%!TEX root = thesis.tex

\chapter{Summary}
\label{chap:summary}

A number of statements were examined throughout the preceding chapters.
-there is a need for increased audience understanding within live coding
-there is a need for increased audience enjoyment within live coding
-there is a need for increased live coder understanding within live coding
-there is a need for increased live coder enjoyment within live coding
-there is a measurable effect on enjoyment with the visualisations compared to without
-there is a measurable effect on understanding with the visualisations compared to without

-visualisations have a place in live coding
-visualisations have a place in understanding the coding process
-visualisations can communicate the coding process more effectively than without
-visualisations have a place in understanding code structures

-qualitative factors seemed to contribute more to the appreciation of the visualisations. People tended towards suggesting that the visualisations should match the music more than the code?

{\color{red} state there validity here}

\section{Contributions}

Throughout the investigation of the application of software visualisations to live coding a number of contributions to the field have been identified. This section outlines these contributions.

Principally, a method of software visualisation has been identified and evaluated. Up to now there has been no attempt to evaluate visualsations within the space of live coding. The method of process-driven visualisation identified has included a combination of static and dynamic code analysis and presentation to audiences during live coding performances.

Through the development of the software visualisation prototypes, a strategy for developing visualisations using a software engineering approach has been identified. This approach has involved collaboration with a musical artist with the application of a software engieering design, iteration and validation approach through a variety of field and user studies.

The development of the software visualisation prototype and the application to the live coding space have allowed the development of a method of evaluating software visualisations within a live performance space. The more general implications of this method of visualisation evaluation may be useful in the evaluation of software developed for large audiences and audiences observing and individual developing software.

% Finally, a novel approach to visualising the programming process has been developed.

Updated software visualisation taxonomy built on the existing software visualisation frameworks...

Method of developing process driven software visualisation for live coding...

Process-driven software visualisation showing software as it is being developed (live, as in live-coding)...

Evaluation of a process-driven software visualisation...

Contribution to live coding process and outcome... A multi-disciplinary approach, implementation and evaluation of a software application.

Discuss recommendations?...

``...methodological issues have to be studied further. This concerns questions like how to design visualizations and how to measure and evaluate the effectiveness of various solutions.''~\cite{VanWijk2005} A methodology for visualisation design has been developed.


\section{Limitations}

Despite significant contributions in the space of visualisation methodology and software visualisation design, development and evaluation, some limitations with the process have been identified.

limitations to evaluation methodlogy
-type of language may affect understanding. Imperitive vs functional etc.

limitations to generalisation
-how wide can the results of this study be generalised outside of the field of live coding.

limitations to assumptions
-understanding and enjoyment are not necessarily entirely independent. enjoyment may contribute to understanding. understanding may contribute to enjoyment.

limitations to visualisation methodology

\section{Future Work}

Software visualisations that interact directly with the programmer are yet to be adopted into mainstream development. Similarly, the visualisation of the live coding process has not yet been adopted into live coding performances.

\subsection{Software Engineering}


\subsection{Live Coding}


\subsection{Evaluation Methodology}

The evaluation methodology described...



Aesthetic elements of the software...

Didactic elements of the software...

Future of visualisations in live coding...

Application to software engineering practice...

Application to the arts...

Visualisation as a means to communicate the coding process...

Visualisation as a means to more effective coding...

More general application to software development


