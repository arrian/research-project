%%
%% Template chap2.tex
%%

\chapter{Survey}
\label{cha:survey}

\section{Purpose}

The purpose of this interview was to gain insight into the audience’s current understanding and enjoyment of the live coding process. Additionally, the relationship between enjoyment and understanding was to be examined. It was hoped that the examination of these factors would further inform the development of visualisations within live coding.

\section{Hypothesis}



\textit{Describe what you think will happen.}
\section{Materials}

-Computer with Extempore
-Live coding performance venue

\textit{List special materials you used.}
\section{Method}
Survey questions were distributed following a live coding performance. Both an online and paper copy were distributed.

\textit{Write a step by step description of what you actually did, identifying the different variables and how you controlled them. Describe what things you changed (variables you manipulated).}
\section{Results}
A total of thirteen survey responses were received. Of these, 77\% regularly listen to music and 54\% perform regularly. 38\% of the respondents have high exposure to programming through work, study or their hobbies, as opposed to 31\% who have no experience with it. Of the respondents, 69\% had never been to a live coding performance before.

Enjoyment was measured according to the relative change in enjoyment through the performance from the beginning to the end. 46\% of survey respondents had high enjoyment throughout the performance. The results for enjoyment are summarised in Table 1. The results suggest an overall high level of enjoyment of the performance. No respondents chose  low enjoyment throughout the performance.


\begin{tabular}{|c|c|c|c|c|c|c|}
\hline \\
Dimension & Flat & High & Low & High to Low & Low to High & Unsure\\
\hline \\
Count & 2 &6 &0 &2 &1 &2\\
\hline
\end{tabular}
Table 1 : Enjoyment through the performance

Similarly, understanding was measured according to the relative change in understanding through the performance from the beginning to the end. 31\% of survey respondents had no change to understanding through the performance   The results for understanding are summarised in Table 2. Overall, understanding is spread out more than enjoyment with only 15\% suggesting that they could understand the relationship between the visuals and the music throughout the performance. There is no statistically significant relationship ($p > .05$) between music listening habits and understanding nor is there a statistically significant relationship ($p > .05$) between coding experience and understanding.

\begin{tabular}{|c|c|c|c|c|c|c|}
\hline \\
Dimension & Flat & High & Low & High to Low & Low to High & Unsure\\
\hline \\
Count & 4
2&
0&
2&
3&
2 \\
\hline
\end{tabular}
Table 2 : Understanding through the performance

The relationship between enjoyment and understanding can be seen in Figure 1. Notably, three respondents who had high enjoyment throughout the performance were the only respondents who had a pattern of low to high understanding. However, the relationship between enjoyment and understanding is not statistically significant ($p > .05$).

69\% of respondents stated that the visuals provided a sense of liveness to the performance. The remained 31\% stated that they had no effect on their sense of liveness. There were no responses stating that the visuals negatively impacted the sense of liveness. 

In terms of confusion, 38\% suggested that no aspects of the visuals were confusing, though 31\% did not respond to the question.

Supplementary observations of the performance are available in Appendix B.


\textit{1.Using all your senses, collect measurable, quantitative raw data and describe what you observed in written form.
2. Reorganise raw data into tables and graphs if you can.
3. Don't forget to describe what these charts or graphs tell us!
4. Pictures, drawings, or even movies of what you observed would help people understand what you observed.}

\section{Discussion}
\textit{1.Based on your observations, what do you think you have learned? In other words, make inferences based on your observations.
2.Compare actual results to your hypothesis and describe why there may have been differences.
3.Identify possible sources of errors or problems in the design of the experiment and try to suggest changes that might be made next time this experiment is done.
4.What have experts learned about this topic? (Refer to books or magazines.)}

\section{Supplementary Observations - remove}
Overall, the live coding performance went smoothly. The room layout was perhaps not optimal for the performance and the display screen was very dim leaving some parts of the source code unreadable. Additionally, the projection surface was not flat further reducing readability.

An estimated 20 people were in attendance at the start of the performance. This grew to an estimated 30 people by the end of the performance, over a time period of about 20 minutes.

The logistics of handing out paper surveys did not suit the venue layout, however most people had the ability to access the survey through their phone. Online survey attrition was about four people, though the accuracy and reason behind this can not be determined from the data. As suggested by Henry, it may be easier just to hand out the QR code on a piece of paper after the performance.

Henry also noted that the other performer used his visuals to tell a story and this story may or may not relate to the music being played.

%%% Local Variables: 
%%% mode: latex
%%% TeX-master: "thesis"
%%% End: 
