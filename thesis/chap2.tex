%%
%% Template chap2.tex
%%

\chapter{Survey}
\label{cha:survey}

\section{Purpose}

A survey was conducted to analyse an audiences existing understanding of the live coding process.

\textit{Describe why you are doing the experiment.}

\section{Hypothesis}



\textit{Describe what you think will happen.}
\section{Materials}

-Computer with Extempore
-Live coding performance venue

\textit{List special materials you used.}
\section{Method}
\textit{Write a step by step description of what you actually did, identifying the different variables and how you controlled them. Describe what things you changed (variables you manipulated).}
\section{Observations}
\textit{1.Using all your senses, collect measurable, quantitative raw data and describe what you observed in written form.
2. Reorganise raw data into tables and graphs if you can.
3. Don't forget to describe what these charts or graphs tell us!
4. Pictures, drawings, or even movies of what you observed would help people understand what you observed.}
\section{Results}
\textit{1.Based on your observations, what do you think you have learned? In other words, make inferences based on your observations.
2.Compare actual results to your hypothesis and describe why there may have been differences.
3.Identify possible sources of errors or problems in the design of the experiment and try to suggest changes that might be made next time this experiment is done.
4.What have experts learned about this topic? (Refer to books or magazines.)}

\section{Supplementary Observations - remove}
-I arrived half an hour early for sound check
-approximately four people were setting up the room for the night
-room layout was not designed for the live coding performance
-display screen had very low contrast. brackets on the projection were unreadable
-the projection screen was not flat further reducing the projected code’s comprehensibility
-at starting time an estimated 20 people were in attendance
-a high proportion of the people appeared to be gallery staff or were involved in the festival
-number of people grew to around 30 by the end of the performance
-many people in the audience were experienced musically
-the performance lasted approximately 20 minutes
-reception was generally positive though the crowd was fairly passive
-the logistics of handing out paper surveys did not suit the venue layout
-mobile phone attrition is estimated up to four people
-an easier method of providing the link would likely have given more results. Henry suggested handing out the qr code and link for people to complete at a later time.
-Henry also noted how the other performer used his visuals to tell a story

-overall, 15 people filled out the survey, one result to be removed due to conflict of interest, one survey result was partially invalid leaving a total of 13 survey results
-of the original 15 results, 4 were paper surveys and 11 were online

%%% Local Variables: 
%%% mode: latex
%%% TeX-master: "thesis"
%%% End: 
