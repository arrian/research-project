%%
%% Template abstract.tex
%%

\chapter*{Abstract}
\label{cha:abstract}
\addcontentsline{toc}{chapter}{Abstract}

This thesis describes an empirical study of source code visualisation
as a means to communicate the programming process in ``live coding''
computer music performances. Following an exploratory field study
conducted during a live coding performance at an arts festival, two
different interaction-driven visualisation techniques were
incorporated into a live coding system. We then performed a more
controlled lab study to evaluate the visualisations' contributions
to the audience experience, with emphasis on the (self-reported)
experiential dimensions of \emph{understanding} and
\emph{enjoyment}. Both software visualisation techniques enhanced
audience enjoyment, while the effect on audience understanding was
more complex. We conclude by suggesting how these visualisation
techniques may be used to enhance the audience experience of live
coding.




% Live coding is a method of performance that presents audio and visual content to audiences through programming. Often \"showing the code\" is a fundamental part of the performance in order to retain the attention of the audience and provide a measure of authenticity.

% Currently missing within the research within live coding is a visualisation of the code that represents the artists intent. Previous visualisation techniques present an abstract and often disjoint representation from the associated code. Missing within this context is a formal analysis of how to best represent the artist's intent visually and a formal analysis of the target audience.

%%% Local Variables: 
%%% mode: latex
%%% TeX-master: "thesis"
%%% End: 
