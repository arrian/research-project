%!TEX root = thesis.tex

\chapter{Conclusion}
\label{chap:conclusion}

From presentation

-  coming back to the research question:
	``can the application of visualisation techniques to live coding enhance audience experience?''

 two dimensions of enhancement have been investigated. 
 both dimensions show improvement with visualisations over source code displayed without visualisations.
visualisations show consistent improvement to understanding for those with lower understanding than without the aid of them 
takes an initial warm-up period for the audiences with higher understanding to get an idea if what is going on within the visualisation… gained back through the performance

have we enhanced audience experience?
yes, the results of these studies suggest that this has been the case
nevertheless, this project has shown only the very beginning of what can be achieved by applying visualisations to live coding.
the first steps into the empirical study of visualising live coding have begun


From OzCHI paper

In this first empirical study of audience perception of code visualisation in live coding, we have identified an opportunity for real-time code visualisations to help improve the audience experience of a live coding computer music performance. With few exceptions, our initial survey of a live coding performance at an arts festival revealed a generally low to medium level of audience understanding throughout that performance (although almost half the survey respondents indicated a high level of enjoyment throughout).

In a subsequent lab study, a comparison of two prototype code visualisations indicated that both visualisations seemed to help with enjoyment. Significantly more audience members reported that our didactic visualisations helped with understanding but overall trends for both enjoyment and understanding throughout the performances were complex. There are indications of a higher cognitive load for the didactic visualisations than the aesthetic visualisations and this may have influenced audience responses to them.

In a future extension of this work, design lessons from both visualisation types could be combined together to produce live coding driven visualisations which targeted both the aesthetics as well as a greater understanding of the live coding process. These visualisations could then be compared with the baseline ``no visualisation'' condition in an audience experiment. There are also opportunities to vary the nature of the visualisations over the course of a performance.

-from discussion with henry:
	-what is the `ambition' of live coding
	-where is the potential for visualisations in live coding

-from discussion with ben:
	-base all future statements on the literature/voices of the influential live coders.

-restate research question

-discuss how it has been examined

-discuss how it has been answered

Discuss wider implications of results
-where could the results of this be applied effectively 

% Questions:
% (1) What about the multimedia arts domain? What about the application of aesthetics and consideration for design within the space of software engineering? Both are widely accepted areas

% (2) Why don't we even have the most simple of tools to visualise changing code structure? Too complex? Not helpful? Why must code structures be designed statically?

% (3) Where is the future of programming? It is esentially the same as it was when it was first created. (Focus on higher abstractions in some fields? Move towards less abstraction...) Yet today it claims to be interactive/responsive etc...

Final words
-bring it back to the f brooks quote from introduction... 
% What was the thesis about?
% What did I find out?

