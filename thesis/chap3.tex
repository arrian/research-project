%!TEX root = thesis.tex

\chapter{Visualisation Design}
\label{chap:visualisation-design}

Following the field study (see Chapter~\ref{chap:exploratory-field-study}), a strategy for visualisation prototyping and evaluation was developed. The field study indicated a need for further investigation of the application of visualisations to the live coding process. Further reasoning behind developing visualisations, the process and the resulting software prototype are outlined in this chapter.

% {\color{red}\cite{Ware2013a,McLean2010a,Purchase1996} will be useful here.}

\section{Rationale}

Evaluation of the literature identified the need for visualisations to be developed to help live coding audiences understand and enjoy the live coding performance.

% Two code visualisations were developed in a live coding context to determine effective presentational and educational features. Two visualisations were evaluated including a visualisation targeting aesthetic appeal and a visualisation with a more didactic approach. The goal was to determine the usability, differences and desirability of the two approaches to further inform future live coding visualisations.\\

% The set of didactic visualisations predominantly focussed on the relationship between the live coding active processes and their behaviour. The visualisations prominently displaying the names of the active functions with visual indication of the number of functions running and their callback time. Bright colours and solid shapes were used to ensure constant visibility and communicate the intention of the underlying code. Overall, four visualisations were presented with each introduced depending on the number of active functions. It was predicted that taking a more educational approach would see a reduction in audience confusion through the performance.\\

% The set of aesthetic visualisations focussed less on the programmatic aspects of the live coding performance, rather intending to provide additional visual interest to the projected code thereby prolonging attention. More variety was used in visual structure and colour. Again, four visualisations were presented, varying the visualisation based on the number of active functions. It was predicted that focussing on the aesthetic nature of the visualisations would assist in audience retention and result in a consistency of interest through the performance.

\section{Requirements}

In order to systematically and efficiently develop a visualisation design strategy, two stakeholders were identified.

The audience was the intended audience and was identified as a critical stakeholder in the ultimate effectiveness of the visualisations developed. The audience evaluated in the initial field study would provide 


% Avoid interfering with the code too much...
% Using the field study as a type of requirements gathering... what did we learn from the field study?
% 


\section{Risks}


In the early stages of the development of a visualisation strategy, a number of risks were identified. A selection of these risks and the associated risk management strategies are outlined below.

Fundamental to live coding is the performance element. Live coding is usually presented to a large audience and would require visualisations that did not cause any failure during the performance. This risk was identified as the most important to mitigate. A test strategy was developed for the visualisations to ensure that they did not cause unrecoverable failures and no memory leaks.

Timeline constraints were also identified as a potential risk factor in the development and evaluation of the visualisations. Mitigation involved developing a flexible, iteration based plan with estimated milestones. {\color{red} Add timeline in here......}

-risk that the visualisations would not provide any benefit or would negatively impact the performance... mitigated by the iteration based approach to developing the visualisations

-risk that the visualisations would interfere with the live coder...

% -risk that the visualisations were not the correct strategy to begin with?

% Other risks?

\section{Design}

The design of the initial visualsations focussed on the application of the literature on visualisation design to the field of software visualisation and the application to the visualisation of live code.

-total of 8 visualisation progressions were developed
-4 progressions for both didactic and aesthetic

-guidelines set out by mclean were used in understanding the application of visualisation to live coding...
\cite{McLean2010a}...

-guidelines set out by ware were used in developing the visualisations...
% - what are the guidelines used
\cite{Ware2013a}...


\begin{figure}
  \centering \includegraphics[width=\columnwidth]{../images/diagrams/knowledge-flow-initial.pdf}
  \caption{Knowledge flow from programmer to observer as directed by the visualisation technique employed.}
\label{fig:knowledge-flow-initial}
\end{figure}


\afterpage{
\begin{figure}
\centering
\includegraphics[width=0.75\columnwidth]{../study-2/results/visualisations/didactic-vis-overlay}
\caption{An example didactic visualisation.}
\label{fig:didactic-visualisation}
\end{figure}

\begin{figure}
\centering
\includegraphics[width=0.75\columnwidth]{../study-2/results/visualisations/aesthetic-vis-overlay}
\caption{An example aesthetic visualisation.}
\label{fig:aesthetic-visualisation}
\end{figure}
\clearpage}

Visualisations developed employed dynamic analysis of the running code to generate the visuals. The intended knowledge flow from programmer to observer is shown in Figure~\ref{fig:knowledge-flow-initial}.

Music visualisation is an extremely rich and open-ended task, so to guide the development of the visualisations for our lab study, we used the concepts of understanding and enjoyment from the initial survey to develop two new code visualisations: a \emph{didactic} one and an \emph{aesthetic} one. 

\subsection{Didactic Visualisation}
\label{sec:didactic-visualisation}

The didactic visualisation (shown in Figure~\ref{fig:didactic-visualisation}) attempted to communicate \emph{information} about the actions of the programmer, prominently displaying the \emph{names} of the active (source code) functions and the ``time until next execution'' for each function (which is particularly relevant in a time-sensitive programming context such as music making). Bright colours and solid shapes were used to ensure constant visibility and to communicate the intention of the underlying code. The didactic visualisations proceeded through four stages, with phase changes made depending on the number of active functions (instruments).

\subsection{Aesthetic Visualisation}
\label{sec:aesthetic-visualisation}

The aesthetic visualisation technique was designed to react to the programmer's activity in a more abstract way, to maximise aesthetic appeal~\cite{Cawthon2007} and to engage the audience's interest. Although still based on the source code and the livecoder's edits, the generation of shapes was driven by instrument volume and synchronised with the musical beat. The emphasis for the aesthetic visualisation was on the artistic appeal of the visuals (see Figure~\ref{fig:aesthetic-visualisation}), including more variety in visual structure and colour. As in the didactic condition, the aesthetic visualisations proceeded through four stages, based on the number of active functions (instruments), but these visuals had no textual labels and they moved and interacted with each other over the entire projected scene.

\subsection{Mappings}

Function count...

Performance stage...

Sound volume contribution by function...

Beat by function...

\section{Summary}

 


