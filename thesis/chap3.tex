%%
%% Template chap3.tex
%%

\chapter{Follow-Up Interview}

\section{Purpose}
The purpose of the follow-up interviews was to gain an in-depth view of what some of the audience believed they would like to understand more about the performance.

Additionally, it was anticipated that some audiences may prefer not to understand more about the technical details of the performance but would be more interested in focussing on the music throughout the performance. The extent of this sentiment was to be examined.

\section{Method}
Two questions were asked of three audience members a number of days after a performance. These two questions were:
What did you \'understand\' about what was going on with the code being projected? In particular, what did you understand about the relationship between the code and the music?
Would would you like to understand more about the code in order to enjoy the performance more?

\section{Results and Discussion}
A total of three responses were received. interviews are available in Appendix A.

All three respondents referred to the initial stage of developing loops to be played and two of these referred to the silence preceding the music suggesting an understanding within the initial stages of the performance. Two interviewees mentioned that they recognised changes to the code and changes to the music separately but could not see the link between the two.

All three interviewees suggested that it would be nice to know a little about the code or the concepts but it is not essential and may even be detrimental to enjoying the music. One interviewee mentioned switching between observing the code and listening to the music. This transition was described as \'switching off\' suggesting that the code takes mental effort to understand whereas the music does not.