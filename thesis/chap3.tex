%!TEX root = thesis.tex

\chapter{Exploratory Field Study}
\label{cha:survey}

After a live coding performance at the ``You Are Here'' arts festival in Canberra, audience members were asked to fill out a survey regarding their perception of
and response to the projection of the computer code during the
performance. Each audience member was asked to indicate which of a
number of curves/trajectories best represented their \emph{enjoyment}
and \emph{understanding} of the performer's actions in typing the code
through the performance. These trajectories allowed for ``high'',
``medium'', and ``low'' levels of enjoyment/understanding for the
(self-determined) ``beginning'', ``middle'' and ``end'' of the
performance. Other survey questions addressed their sense of
``liveness'' of the performance (c.f.~\cite{Auslander}) and whether
the projected code was confusing.

\section{Rationale}
\label{sec:fieldstudyrationale}

The purpose of this interview was to gain insight into the audience’s current understanding and enjoyment of the live coding process. Additionally, the relationship between enjoyment and understanding was to be examined. It was hoped that the examination of these factors would further inform the development of visualisations within live coding.

\section{Method}
\label{sec:fieldstudyprocedure}

Survey questions were distributed following a live coding performance. Both an online and paper copy were distributed.

\textit{Write a step by step description of what you actually did, identifying the different variables and how you controlled them. Describe what things you changed (variables you manipulated).}


\section{Participants}
\label{sec:fieldstudyparticipants}



\section{Results}
\label{sec:fieldstudyresults}

%%%%%%%%%%%Text from OzCHI paper

Of the thirteen survey responses received, six audience members showed
a high level of enjoyment throughout the whole performance, while the
remaining seven responses showed alternating levels of enjoyment. No
audience members indicated a low level of enjoyment throughout the
performance.

Only two of the thirteen respondents indicated that they understood
the relationship between the code projections and the music throughout
the performance. Three of the six respondents who reported a high
level of enjoyment throughout the performance also indicated an
increase in understanding (from low to high) as the performance
progressed, although a Chi-square analysis revealed no significant
relationship between enjoyment and understanding due to the small
sample size. Nine of the thirteen respondents stated that the code
projection provided a sense of liveness to the performance and the
remainder stated that viewing the code had no effect on their sense of
liveness. Four respondents felt that the code projections were
confusing, five felt that they were not confusing, and four did not
answer the question.

Taken as a whole, the results of this small field study were
salutatory towards the benefit of ``seeing as well as hearing'' code
during a live coding performance, especially as far as the general
public is concerned. The majority of the audience felt that they made
the performance seem more ``live''. However, a minority stated that
they found the projections confusing and only a very small number of
respondents claimed to have actually understood what the programmer
was doing. We were quite intrigued by the small cohort of respondents
whose understanding increased through the performance and whose
enjoyment remained high, and we wished to test whether augmenting code
projections with additional visualisations might increase the
understanding and enjoyment of the audience in live coding.

%%%%%%%%%%%%%%%%%%%%%%%%%%% Initial Report Text

A total of thirteen survey responses were received. Of these, 77\% regularly listen to music and 54\% perform regularly. 38\% of the respondents have high exposure to programming through work, study or their hobbies, as opposed to 31\% who have no experience with it. Of the respondents, 69\% had never been to a live coding performance before.

Enjoyment was measured according to the relative change in enjoyment through the performance from the beginning to the end. 46\% of survey respondents had high enjoyment throughout the performance. The results for enjoyment are summarised in Table 1. The results suggest an overall high level of enjoyment of the performance. No respondents chose  low enjoyment throughout the performance.\\


\begin{tabular}{|c|c|c|c|c|c|c|}
\hline 
Dimension & Flat & High & Low & High to Low & Low to High & Unsure\\
\hline 
Count & 2 &6 &0 &2 &1 &2\\
\hline
\end{tabular}\\
Table 1 : Enjoyment through the performance

Similarly, understanding was measured according to the relative change in understanding through the performance from the beginning to the end. 31\% of survey respondents had no change to understanding through the performance   The results for understanding are summarised in Table 2. Overall, understanding is spread out more than enjoyment with only 15\% suggesting that they could understand the relationship between the visuals and the music throughout the performance. There is no statistically significant relationship ($p > .05$) between music listening habits and understanding nor is there a statistically significant relationship ($p > .05$) between coding experience and understanding.\\

\begin{tabular}{|c|c|c|c|c|c|c|}
\hline 
Dimension & Flat & High & Low & High to Low & Low to High & Unsure\\
\hline 
Count & 4
2&
0&
2&
3&
2 \\
\hline
\end{tabular}\\
Table 2 : Understanding through the performance

The relationship between enjoyment and understanding can be seen in Figure 1. Notably, three respondents who had high enjoyment throughout the performance were the only respondents who had a pattern of low to high understanding. However, the relationship between enjoyment and understanding is not statistically significant ($p > .05$).

69\% of respondents stated that the visuals provided a sense of liveness to the performance. The remained 31\% stated that they had no effect on their sense of liveness. There were no responses stating that the visuals negatively impacted the sense of liveness. 

In terms of confusion, 38\% suggested that no aspects of the visuals were confusing, though 31\% did not respond to the question.

Supplementary observations of the performance are available in Appendix B.

\textit{1.Using all your senses, collect measurable, quantitative raw data and describe what you observed in written form.
2. Reorganise raw data into tables and graphs if you can.
3. Don't forget to describe what these charts or graphs tell us!
4. Pictures, drawings, or even movies of what you observed would help people understand what you observed.}

\section{Discussion}
\label{sec:fieldstudydiscussion}

\textit{1.Based on your observations, what do you think you have learned? In other words, make inferences based on your observations.
2.Compare actual results to your hypothesis and describe why there may have been differences.
3.Identify possible sources of errors or problems in the design of the experiment and try to suggest changes that might be made next time this experiment is done.
4.What have experts learned about this topic? (Refer to books or magazines.)}



% \chapter{Follow-Up Interview}

% \section{Purpose}
% The purpose of the follow-up interviews was to gain an in-depth view of what some of the audience believed they would like to understand more about the performance.

% Additionally, it was anticipated that some audiences may prefer not to understand more about the technical details of the performance but would be more interested in focussing on the music throughout the performance. The extent of this sentiment was to be examined.

% \section{Method}
% Two questions were asked of three audience members a number of days after a performance. These two questions were:
% What did you \'understand\' about what was going on with the code being projected? In particular, what did you understand about the relationship between the code and the music?
% Would would you like to understand more about the code in order to enjoy the performance more?

% \section{Results and Discussion}
% A total of three responses were received. interviews are available in Appendix A.

% All three respondents referred to the initial stage of developing loops to be played and two of these referred to the silence preceding the music suggesting an understanding within the initial stages of the performance. Two interviewees mentioned that they recognised changes to the code and changes to the music separately but could not see the link between the two.

% All three interviewees suggested that it would be nice to know a little about the code or the concepts but it is not essential and may even be detrimental to enjoying the music. One interviewee mentioned switching between observing the code and listening to the music. This transition was described as \'switching off\' suggesting that the code takes mental effort to understand whereas the music does not.





% \section{Observations}
% Overall, the live coding performance went smoothly. The room layout was perhaps not optimal for the performance and the display screen was very dim leaving some parts of the source code unreadable. Additionally, the projection surface was not flat further reducing readability.

% An estimated 20 people were in attendance at the start of the performance. This grew to an estimated 30 people by the end of the performance, over a time period of about 20 minutes.

% The logistics of handing out paper surveys did not suit the venue layout, however most people had the ability to access the survey through their phone. Online survey attrition was about four people, though the accuracy and reason behind this can not be determined from the data. As suggested by Henry, it may be easier just to hand out the QR code on a piece of paper after the performance.

% Henry also noted that the other performer used his visuals to tell a story and this story may or may not relate to the music being played.



%%% Local Variables: 
%%% mode: latex
%%% TeX-master: "thesis"
%%% End: 
