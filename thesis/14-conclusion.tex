%!TEX root = thesis.tex

\chapter{Conclusion}
\label{chap:conclusion}

This thesis has investigated the proposition that ``code visualisations improve the experience of observers''. The studies carried out indicate that this may be the case and that there is potential for the further development of visualisations within this space. The application of code visaulisations to live coding and the subsequent evaluation demonstrated that, even within the arts, a scientific analysis sensitive to the artistic goals could be conducted. 

In this first empirical study of audience perception of code visualisation in live coding, we have identified an opportunity for real-time code visualisations to help improve the audience experience of a live coding computer music performance. Nevertheless, this examination is the first in a series of steps in the systematic and empirical evaluation of live coding visualisation. There is still significant opportunity to define the direction of visualisations within live coding.


% -restate research question
% More specifically, this thesis investigates the question: ``can the application of visualisation techniques to live coding enhance audience experience by increasing understanding and enjoyment?''. These questions were examined through a process of prototype development, user study evaluation and refinement.

% -state that there is opportunity within live coding to define the direction of visualisations
% % In this first empirical study of audience perception of code visualisation in live coding, we have identified an opportunity for real-time code visualisations to help improve the audience experience of a live coding computer music performance. With few exceptions, our initial survey of a live coding performance at an arts festival revealed a generally low to medium level of audience understanding throughout that performance. In a subsequent lab study, a comparison of two prototype code visualisations indicated that both visualisations seemed to help with enjoyment. Significantly more audience members reported that our didactic visualisations helped with understanding but overall trends for both enjoyment and understanding throughout the performances were complex. There are indications of a higher cognitive load for the didactic visualisations than the aesthetic visualisations and this may have influenced audience responses to them.

% -state that this study has shown that the arts can be examined from a scientific perspective
% % Throughout this thesis, the application of visualisations to live coding to enhance audience experience has been examined. Two dimensions of understanding and enjoyment were identified as effective means of evaluating these visualisations. Potential for visualisations was identified with both dimensions of understanding and enjoyment showing improvement with visualisations displayed with source code than without.

% % -where is the potential for visualisations in live coding or evaluation of visualisations within live coding?
% % Nevertheless, this examination is the first in a series of steps in the systematic and empirical evaluation of live coding visualisation. Results suggest that visualisations have much potential for improvement and much potential to enhance the audience's experience.

% -what is the ambition of live coding?

% -given this study, where will the visualisation of live code go in the future?
% % live coding is still to be fully defined as a practice
% % looks to show the screen -- elucidate the programmer

% -where could the results of these studies be applied effectively?

% % -what do the influential live coders say?

% -what are the wider implications of the research to science and the arts?
% % art and science perspective

% -final words
% quote?

