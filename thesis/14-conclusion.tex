%!TEX root = thesis.tex

\chapter{Conclusion}
\label{chap:conclusion}

-restate research question
% This thesis investigates the proposition that ``code visualisation improves the experience of observers'' in the setting of a live arts practice. More specifically, this thesis investigates the question: ``can the application of visualisation techniques to live coding enhance audience experience by increasing understanding and enjoyment?''. These questions were examined through a process of prototype development, user study evaluation and refinement.

-discuss how it has been examined
% In this first empirical study of audience perception of code visualisation in live coding, we have identified an opportunity for real-time code visualisations to help improve the audience experience of a live coding computer music performance. With few exceptions, our initial survey of a live coding performance at an arts festival revealed a generally low to medium level of audience understanding throughout that performance. In a subsequent lab study, a comparison of two prototype code visualisations indicated that both visualisations seemed to help with enjoyment. Significantly more audience members reported that our didactic visualisations helped with understanding but overall trends for both enjoyment and understanding throughout the performances were complex. There are indications of a higher cognitive load for the didactic visualisations than the aesthetic visualisations and this may have influenced audience responses to them.


-discuss how it has been answered
% Throughout this thesis, the application of visualisations to live coding to enhance audience experience has been examined. Two dimensions of understanding and enjoyment were identified as effective means of evaluating these visualisations. Potential for visualisations was identified with both dimensions of understanding and enjoyment showing improvement with visualisations displayed with source code than without.

-where is the potential for visualisations in live coding or evaluation of visualisations within live coding?
% Nevertheless, this examination is the first in a series of steps in the systematic and empirical evaluation of live coding visualisation. Results suggest that visualisations have much potential for improvement and much potential to enhance the audience's experience.

-what is the ambition of live coding?

-where could the results of these studies be applied effectively

-what do the influential live coders say?

-what are the wider implications of the research
% art and science perspective

-final words
