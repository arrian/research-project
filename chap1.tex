%%
%% Template chap1.tex
%%

\chapter{Literature Review}
\label{cha:literaturereview}

Research Project - Article Summaries

A principled approach to developing new languages for live coding
-Focusses on musicians entering the field of live coding
-Discuss domain specific languages (e.g. spreadsheets in accounting)
-Approaching live coding as a way to extend the musician rather than the programmer becoming a musician
-Interfacing with external hardware
-Cognitive ergonomics of language design
-Declarative constraint propagation
-Direct manipulation over indirect manipulation allows audience to perceive relationship between action and effect
-Use supercollider as the live coding platform
-Critical technical practice

Algorithms as Scores: Coding Live Music
-considers live coding as a new branch of musical score
-Kadinsky and Klee representing synchronic process (painting) as diachronic process (music)
-graphical representations of music
-graphical scores as special representation of an algorithm
-Claudia Molitor’s 3D Score Series - engaging with score
-Basically describes the history and modern live coding practice

An Approach to Musical Live Coding
-aa-cell performances
-does remapping a function to a random function produce measurable results
-Overview of the live coding environment and practice

Visual Music Instrument
-synæsthetic composition, computational expression and the dynamics of performance are important research axes
-History of live visual performances
-painted composition is closer to a single musical instance than it is to musical composition.
-music is abstract, visuals are moving that direction too
-Visual Music Instrument Design
-

A Principled Taxonomy of Software Visualisation
-Discusses visualisation of algorithm
-level of abstraction vs level of animation
-aspect vs abstractness vs animation vs automation
-data vs code
-static vs animated
-No demonstrable gains from software visualisation seen
-Very old article

A Model-Based Visualisation Taxonomy
-scientific visualisation vs information visualisation
-model based visualisation taxonomy divides groups into continuous and discrete models
-continuous model divides into 3 dimensions including dependent variables, data type, and number of independent variables
-discrete model divides into connected and unconnected data types

 A Taxonomy of Glyph Placement Strategies for Multidimensional Data Visualisation
-shows taxonomy of glyph placement strategies
-not immediately relevant

A Taxonomy of Program Visualisation Systems
-Scope (Code, Data state, Control state, Behaviour)
-Abstraction Level (Direct representation, Structural representation, synthesised representation)
-Specification method (Predefinition, Annotation, Declaration, Manipulation)
-Interface (Simple objects, Composite objects, visual events, dimensionality, multiple worlds, control interaction, image interaction)
-Presentation (Analytical, Explanatory, Orchestration)

Improvising Synesthesia
-introduction of the term comprovisation
-has been no visual creative process in which the artistic process is available to the audience
-improvisation of visual art
-

Live Coding Towards Computational Creativity
-describes what live coding is and potential future directions in terms of computational creativity
-includes live coder survey (http://doc.gold.ac.uk/∼ma503am/writing/icccx/
-

Painterly Interface for Audiovisual Performance
-Describes the history of audio visual performance (incl. Castel’s Ocular Harpsichord, Thomas Wilfred’s Clavilux, Oskar Fischinger’s Lumigraph, Charles Dockum’s MobilColor Projector, )

AVVX - A Vector Graphics Tool for Audiovisual Performances
-Survey for ease of use and utility of the AVVX engine
-

Content-based Mood Classification for Photos and Music
-Variety of emotion classifications:
	-Thayer’s model: stress vs energy
	-Russell’s model: pleasantness vs alertness
	-Tellegen-Watson-Clark model: positive affect vs negative affect
	-Reisenzeins model: pleasantness vs alertness
-Classified images and music as: aggressive, euphoric, melancholic, calm
-Found that a combination of dimensional models and category-based models provided the most useful results

Dimensions in Program Synthesis
-Three dimensions in program synthesis: User intent, search space (expressiveness vs efficiency), search technique (eg. brute-force)

Dimensions of Software Architecture for Program Understanding
-Three dimensions of software architecture that affect user involvement: level of abstraction, degree of domain specific knowledge, degree of automation

Gathering Audience Feedback on an Audiovisual Performance
-Modes of engagement: Perceptive, interpretive and reflective

The Programming Language as a Musical Instrument
-Discusses differences between software engineering and live coding as a musical practice
-Utilitarian design focus helps live coders see beyond the narrow focus of live coding performance itself and see the underlying software engineering focus including requirements analysis, design, reuse, debugging, maintenance etc.

Rethinking Visualization: A High-Level Taxonomy
-Old method of categorisation: Scientific vs Information (incl. factors such as scientific vs non-scientific, physical vs abstract, spacialisation given vs specialisation chosen)
-Introduce 'model based' visualisation techniques
-“our main objective is to provide insight into how different research areas relate, not to provide guidelines for visualisation design."
-Terminology - Object of study, Data, Design model, User model (see image in article for relationship)
-Taxonomy developed within this article consists of {discrete, continuous} vs display attributes (eg. given, constrained, chosen) per the design model
-Continuous model visualisation is broken down according to the number if independent and dependent variables and the type of the dependent variables (incl. scale, vector and tensor)
-Discrete model visualisations are broken down into wether the data structure or data values are visualised

-Articles discusses lower level taxonomy including - spacial relationships, numeric trends, patterns, connectivity, filtering
-

Heuristics for Information Visualization Evaluation
-suggests that between 3 and 5 heuristics for evaluation would be enough
-discusses heuristics including:
	-
	-
	-
	-

\section{Live Coding}
\label{sec:livecoding}

(focus on developing a narrative concerning what needs to be done within live coding to achieve the software engineering goals and what needs to be done to develop successful visualisations within the field of live coding)
\\\\
Live coding describes the process of exposing the programming process to a live audience. -talk more about what live coding is
\\\\
Live coding history... . -talk more about history of live coding
\\\\
There exists much discussion within the live coding research body (eg. ...) about the potential for live visual manipulation and examination of the current progress within the field to achieve this.


\subsection{Music vs Visualisation}

In addition, there has been a move towards manipulation of the visuals in synchronisation with the ....


\section{Music Visualisation}
\label{sec:musicvisualisation}

\subsection{Taxonomy}

\subsection{Live Performance}


\section{Software Engineering Practice}
\label{sec:softwareengineering}

As the field of live coding develops, the relevance of both the application of software engineering practice to the field and the relevance of live coding to the field of software engineering has become highly apparent...

\subsection{Application of Software Engineering to Live Coding}
The application of software engineering to live coding...
(\cite{Blackwell2005} paper incl. requirements analysis, design, coding, project management, reuse, debugging, documentation, comprehension and maintenance)

\subsubsection{Design}
Design...
\subsubsection{Coding}
Coding...
\subsubsection{Comprehension}
Comprehension...

\subsection{Application of Live Coding to Software Engineering}
The application of live coding to software engineering...

\subsubsection{Dissemination of Code Understanding}


\subsubsection{Multidisciplinary Cohesion}

\subsubsection{Visualisation Framework}





%%% Local Variables: 
%%% mode: latex
%%% TeX-master: "thesis"
%%% End: 
