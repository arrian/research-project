%%
%% Template chap1.tex
%%

\chapter{Literature Review}
\label{cha:literaturereview}

Research Project - Article Summaries

A principled approach to developing new languages for live coding
\begin{itemize}
\item Focusses on musicians entering the field of live coding
\item Discuss domain specific languages (e.g. spreadsheets in accounting)
\item Approaching live coding as a way to extend the musician rather than the programmer becoming a musician
\item Interfacing with external hardware
\item Cognitive ergonomics of language design
\item Declarative constraint propagation
\item Direct manipulation over indirect manipulation allows audience to perceive relationship between action and effect
\item Use supercollider as the live coding platform
\item Critical technical practice
\end{itemize}

Algorithms as Scores: Coding Live Music
\begin{itemize}
\item considers live coding as a new branch of musical score
\item Kadinsky and Klee representing synchronic process (painting) as diachronic process (music)
\item graphical representations of music
\item graphical scores as special representation of an algorithm
\item Claudia Molitor’s 3D Score Series - engaging with score
\item Basically describes the history and modern live coding practice
\end{itemize}

An Approach to Musical Live Coding
\begin{itemize}
\item aa-cell performances
\item does remapping a function to a random function produce measurable results
\item Overview of the live coding environment and practice
\end{itemize}

Visual Music Instrument
\begin{itemize}
\item synæsthetic composition, computational expression and the dynamics of performance are important research axes
\item History of live visual performances
\item painted composition is closer to a single musical instance than it is to musical composition.
\item music is abstract, visuals are moving that direction too
\item Visual Music Instrument Design
\end{itemize}

A Principled Taxonomy of Software Visualisation
\begin{itemize}
\item Discusses visualisation of algorithm
\item level of abstraction vs level of animation
\item aspect vs abstractness vs animation vs automation
\item data vs code
\item static vs animated
\item No demonstrable gains from software visualisation seen
\item Very old article
\end{itemize}

A Model-Based Visualisation Taxonomy
\begin{itemize}
\item scientific visualisation vs information visualisation
\item model based visualisation taxonomy divides groups into continuous and discrete models
\item continuous model divides into 3 dimensions including dependent variables, data type, and number of independent variables
\item discrete model divides into connected and unconnected data types
\end{itemize}

 A Taxonomy of Glyph Placement Strategies for Multidimensional Data Visualisation
 \begin{itemize}
\item shows taxonomy of glyph placement strategies
\item not immediately relevant
\end{itemize}

A Taxonomy of Program Visualisation Systems
\begin{itemize}
\item Scope (Code, Data state, Control state, Behaviour)
\item Abstraction Level (Direct representation, Structural representation, synthesised representation)
\item Specification method (Predefinition, Annotation, Declaration, Manipulation)
\item Interface (Simple objects, Composite objects, visual events, dimensionality, multiple worlds, control interaction, image interaction)
\item Presentation (Analytical, Explanatory, Orchestration)
\end{itemize}

Improvising Synesthesia
\begin{itemize}
\item introduction of the term comprovisation
\item has been no visual creative process in which the artistic process is available to the audience
\item improvisation of visual art
\end{itemize}

Live Coding Towards Computational Creativity
\begin{itemize}
\item describes what live coding is and potential future directions in terms of computational creativity
\item includes live coder survey (http://doc.gold.ac.uk/∼ma503am/writing/icccx/
\end{itemize}

Painterly Interface for Audiovisual Performance
\begin{itemize}
\item Describes the history of audio visual performance (incl. Castel’s Ocular Harpsichord, Thomas Wilfred’s Clavilux, Oskar Fischinger’s Lumigraph, Charles Dockum’s MobilColor Projector, )
\end{itemize}

AVVX - A Vector Graphics Tool for Audiovisual Performances
\begin{itemize}
\item Survey for ease of use and utility of the AVVX engine
\end{itemize}

Content-based Mood Classification for Photos and Music
\begin{itemize}
\item Variety of emotion classifications:
\begin{itemize}
	\item Thayer’s model: stress vs energy
	\item Russell’s model: pleasantness vs alertness
	\item Tellegen-Watson-Clark model: positive affect vs negative affect
	\item Reisenzeins model: pleasantness vs alertness
	\end{itemize}
\item Classified images and music as: aggressive, euphoric, melancholic, calm
\item Found that a combination of dimensional models and category-based models provided the most useful results
\end{itemize}

Dimensions in Program Synthesis
\begin{itemize}
\item Three dimensions in program synthesis: User intent, search space (expressiveness vs efficiency), search technique (eg. brute-force)
\end{itemize}

Dimensions of Software Architecture for Program Understanding
\begin{itemize}
\item Three dimensions of software architecture that affect user involvement: level of abstraction, degree of domain specific knowledge, degree of automation
\end{itemize}

Gathering Audience Feedback on an Audiovisual Performance
\begin{itemize}
\item Modes of engagement: Perceptive, interpretive and reflective
\end{itemize}

The Programming Language as a Musical Instrument
\begin{itemize}
\item Discusses differences between software engineering and live coding as a musical practice
\item Utilitarian design focus helps live coders see beyond the narrow focus of live coding performance itself and see the underlying software engineering focus including requirements analysis, design, reuse, debugging, maintenance etc.
\end{itemize}

Rethinking Visualization: A High-Level Taxonomy
\begin{itemize}
\item Old method of categorisation: Scientific vs Information (incl. factors such as scientific vs non\item scientific, physical vs abstract, spacialisation given vs specialisation chosen)
\item Introduce 'model based' visualisation techniques
\item “our main objective is to provide insight into how different research areas relate, not to provide guidelines for visualisation design."
\item Terminology - Object of study, Data, Design model, User model (see image in article for relationship)
\item Taxonomy developed within this article consists of {discrete, continuous} vs display attributes (eg. given, constrained, chosen) per the design model
\item Continuous model visualisation is broken down according to the number if independent and dependent variables and the type of the dependent variables (incl. scale, vector and tensor)
\item Discrete model visualisations are broken down into wether the data structure or data values are visualised

\item Article discusses lower level taxonomy including - spacial relationships, numeric trends, patterns, connectivity, filtering
\end{itemize}

Heuristics for Information Visualization Evaluation
\begin{itemize}
\item suggests that between 3 and 5 heuristics for evaluation would be enough
\end{itemize}

\section{Live Coding}
\label{sec:livecoding}

(focus on developing a narrative concerning what needs to be done within live coding to achieve the software engineering goals and what needs to be done to develop successful visualisations within the field of live coding)
\\\\
Live coding describes the process of exposing the programming process to a live audience. -talk more about what live coding is
\\\\
Live coding history... . -talk more about history of live coding
\\\\
There exists much discussion within the live coding research body (eg. ...) about the potential for live visual manipulation and examination of the current progress within the field to achieve this.


\subsection{Music vs Visualisation}

In addition, there has been a move towards manipulation of the visuals in synchronisation with the ....


\section{Music Visualisation}
\label{sec:musicvisualisation}

\subsection{Taxonomy}

\subsection{Live Performance}


\section{Software Engineering Practice}
\label{sec:softwareengineering}

As the field of live coding develops, the relevance of both the application of software engineering practice to the field and the relevance of live coding to the field of software engineering has become highly apparent...

\subsection{Application of Software Engineering to Live Coding}
The application of software engineering to live coding...
(\cite{Blackwell2005} paper incl. requirements analysis, design, coding, project management, reuse, debugging, documentation, comprehension and maintenance)

\subsubsection{Design}
Design...
\subsubsection{Coding}
Coding...
\subsubsection{Comprehension}
Comprehension...

\subsection{Application of Live Coding to Software Engineering}
The application of live coding to software engineering...

\subsubsection{Dissemination of Code Understanding}


\subsubsection{Multidisciplinary Cohesion}

\subsubsection{Visualisation Framework}





%%% Local Variables: 
%%% mode: latex
%%% TeX-master: "thesis"
%%% End: 
