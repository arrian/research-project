%%
%% Template chap2.tex
%%

\chapter{Survey}
\label{cha:survey}

\section{Purpose}

An intial survey was conducted to analyse an audiences existing understanding of the live coding process. 

\textit{Describe why you are doing the experiment.}

\section{Hypothesis}
\textit{Describe what you think will happen.}
\section{Materials}
\textit{List special materials you used.}
\section{Method}
\textit{Write a step by step description of what you actually did, identifying the different variables and how you controlled them. Describe what things you changed (variables you manipulated).}
\section{Observations}
\textit{1.Using all your senses, collect measurable, quantitative raw data and describe what you observed in written form.
2. Reorganise raw data into tables and graphs if you can.
3. Don't forget to describe what these charts or graphs tell us!
4. Pictures, drawings, or even movies of what you observed would help people understand what you observed.}
\section{Results}
\textit{1.Based on your observations, what do you think you have learned? In other words, make inferences based on your observations.
2.Compare actual results to your hypothesis and describe why there may have been differences.
3.Identify possible sources of errors or problems in the design of the experiment and try to suggest changes that might be made next time this experiment is done.
4.What have experts learned about this topic? (Refer to books or magazines.)}

\section{Supplementary Observations - remove}


%%% Local Variables: 
%%% mode: latex
%%% TeX-master: "thesis"
%%% End: 
