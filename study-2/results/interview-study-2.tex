
\section*{Aesthetic Visualisation}

Ben: So the first little synth thing in this I knew pretty much exactly what I wanted to do. I was going to do random pitches at 16th notes. I think I probably did this synth thing exactly the same in other performances.\\

Arrian: Is this a common technique you use?\\

B: Yeah, well. This was not even prepared specifically for this performance but I've done stuff like this in the past. This is a pretty common trick I use for getting up and running early.\\

I think the bass is playing at this point but I can't hear on these speakers.\\

A: How much planning went into this performance?\\

B: This performance was pretty safe. It was pretty simple and pretty preplanned. What I would say is that the form of all of the instruments are pretty standard for my live coding.\\

So for the bass to go through a list of pitches like that... I do that alot.\\

For the drums I play with some sort of modulation of the sample slot, of the kit... I do that alot.\\

I probably choose slightly different parameters each time through but, yeah, none of these things are really adventurous by the standards of my live coding.\\

A: Why is that?\\

B: Honestly, the main reason is that these things work well and I've tried other algorithms and they don't sound as good as these simpler algorithmic structures with judicious choice of parameters.\\

Though I feel that as an artist sure, as a programmer this is not so interesting because the algorithms are pretty safe.\\

There are a lot of people in computer music today that use fancy algorithms, they use cellular automata, they use generative algorithms but I think it works better if you use... I've had more artistic success using these simple algorithms and then just using my musical experience and intelligence to select good parameters...\\

A: Did you plan the performance around the visuals?\\

B: To be honest, I think in the second piece I tried to limit the callback rate because I knew that some of the visuals in the didactic setup worked better with a longer callback rate.\\

A: Do you think the visualisations held you back?\\

B: No, it certainly didn't hold me back. In some ways it is nice to have constraints.\\

I'm now going up an octave. It gives it a harder edge...\\

It is interesting that I'm very conscious of the hypermeter. I'm just grooving along and it just makes sense to evaluate in time.\\

A: Did you find this visualisation distracting?\\

B: Ah, no. In general I'm just so focussed on the code.\\

This little bit is adventurous. This drum bit I didn't know exactly what samples were in those slots. Before I was just guessing some numbers and picking numbers I thought might be good.\\

This bit is certainly more intense. This is more european house. I don't actually know if it is european house but I'm sure there is some name for this genre.\\

This end bit is a bit new. I hadn't planned to end it like that.\\

A: Did it occur by chance?\\

B: Not so much by chance. I got to the end and wasn't really sure how I would finish this. This is true making it up.\\

A: Were you happy with the performance?\\

B: Yeah.\\

I'm actually going diatonically out of the scale I was using for the whole piece.\\

Yeah, I hadn't planned to finish like that but... yeah, yeah, I'm reasonably happy with that.\\

B: I think in terms of the surprising stuff. There were no surprises when I started or added each new instrument. I knew pretty much exactly what I was going to do. I might not have had the exact parameters in mind. I would have put maybe a 60 instead of a 70. Even when I didn't have an exact number in mind I would have had an approximate number in mind... loud vs soft.\\

Once stuff is going then I think all bets are off and I at least don't really think through what I'm going to do after that. Generally I'll go back and start messing with stuff. In that case I went back and messed with the synth.\\

I did a bit of interesting stuff with the drums. I went from a more groovy and pretty standard drum beat to a heightened beat which definitely changed the mood of the piece.\\

I'm not unhappy with that. I'd probably do something different next time just because you do something different every time but that was one of the things that surprised me.\\

In general I was pretty happy with it. I think it is a good sound palette. The drums grooved pretty well which is an important thing. The bass line was pretty cool, though I couldn't hear it in that recording.\\

A: In terms of the visualisations, do you think they added anything to the piece?\\

B: I think they added something. I think they are just ambience. It is definitely cool to have that stuff going on that is a little visually interesting but I wasn't paying attention to them even then. I definitely wasn't paying attention to them on the day. In fact I tuned them out as best I can because I am just trying to focus on the code.\\

Like I was saying before there were times where it was hard to see the code underneath the visualisations.\\

A: Was this during the aesthetic or didactic performance?\\

B: I think it was more the didactic and I think it was the text in the didactic ones and not in the aesthetic.\\

I definitely like the visuals. They definitely add something and they don't take anything away even though I wasn't paying attention to them. I still see the text as the main thing but the visuals are gravy and that was nice gravy.\\

A: The audience was reasonably computer literature. Do you think they would have preferred to focus on the code, the music or the visuals?\\

B: That's a good question. I don't know. I'm really curious.\\

Focus is a funny thing. You rarely explicitly go ``alright, I'm going to focus on blah and focus on blah''. I think you drift. Probably at different points they were paying attention to each.\\

I think I'm a bad judge of what people pay attention to because I pay attention to completely different stuff when I watch it. What I pay attention is probably completely unhelpful in terms of an indicator for what even a computer literate audience member would be paying attention to.\\

\section*{Didactic Visualisation}

Ben: Now this one starts at a slower tempo. Half the speed... 60bpm vs 120bpm.\\

Arrian: Was this due to the nature of the visualisation?\\

B: No, that was just to be different. They both work fine with the visualisations. In fact the visualisation pretty much work with whatever tempo.\\

So this one is a slower starting one. I still get it going fairly quickly.\\

A: Did the visualisations get in the way here?\\

B: It wasn't too bad because it's not over the top of where I was trying to work.\\

This is one of the things that I did have the visuals in mind when I put together this thing. Initially you have the fast spinning visuals. I knew that I was going to slow this one down and go for two bars of eight beat long sustained chords. I knew that that would look cool as a slowly rotating thing.\\

In fact I stuffed it up there. It wasn't so much a typo as I did a tricky thing where I tried to have a couple of overlapping temporal recursions and filter out only the fast one keeping the slow one going. But that relies on changing the code once you've got it to the state you want.\\

I think in general with these parameters I was just messing around. I knew the general form.\\

I've got a couple of polyrhythms. I really like this bit. I think it works well.\\

A: Despite the timing issue with the visualisations?\\

B: Timing is off by half. We knew that was a problem. I still think it works pretty well. I think this one has real potential but it is disappointing that they didn't sync up.\\

This one is just grooving. I quite like the beat in this one.\\

A: Did the visuals affect your ability to see here?\\

B: I can't remember to be honest. It wasn't a big problem to be honest. It probably only happened once through all four pieces.\\

A: Were you tuning out these visuals during the performance?\\

B: Even when I'm watching it I'm tuning out the visuals but definitely during the performance.\\

I don't think this was planned. It is a fairly standard part of my live coding toolbox. I'm just changing the pitch. So it's to do with the bass. Quick ones and then long ones.\\

Then I changed the offset of the chords and the chords would come in staggered. I don't know how well this one worked in the end. I like bits of it but...\\

This one I think the stuff that is kind of up beat and is really groovy is easier to do than this.\\

This bit was disappointing. I had a cool ending in mind and the I stuffed it up here. I forgot to put the tick symbol.\\

A: Did you manage to pull off the cool ending in the second performance?\\

B: No, I tried to do the same thing and stuffed it up in the exact same way. That was interesting and frustrating. I had a cool ending that I just thought of that day where I was going to do some harmonic organ-y stuff, take it through the circle of fifths and do an interesting chord progression but really kind of draw one out for a slow finish. I just forgot to quote that symbol when I went from the minor to the major.\\

If I had other instruments covering me I could of started it again but since everything had died if I started it again it would have been really obvious so I decided in the moment that that is where I would finish it whereas I had one more minute planned. I started to go down a path where I had a minute more of material to finish it off but then just dogged it. Frustratingly I did not quite the same mistake but a similar sort of mistake in the second one. It's kind of really rare... obviously I make typos but I don't tend to make them in that way.\\

A: Was there some reason for the mistakes?\\

B: Not really.\\

A: Chance?\\

B: Yeah, just chance. Just life.\\

A: Was the main goal of this performance to entertain the audience... beyond the research?\\

B: Yeah, I think so. I always want the people to enjoy themselves. I tried to keep them pretty short. I think every little set was under ten minutes. If you're going to do a live coding set longer than ten minutes it needs to be bloody good. So in general I try to stay under ten minutes.\\

It's interesting, some of my earlier videos are longer than ten minutes and I watch them now and I think `this drags on'. I'm much better at it now and I'm much better at making things happen quickly. I'm especially better at getting stuff up and running, partially because I'm an emacs guru and have all the snippet magic to make that happen but also you just learn the little extempore tricks and the general tricks for getting things up and running.\\

For example in the aesthetic set, there was probably stuff going after only 10 seconds. For this one there was probably stuff up before 30 seconds. I reckon you've got to get something up before 30 seconds.\\

A: Was boredom getting to you?\\

B: Not really. Certainly not in a big way. By the fourth I was like ``I'm done, this has been a lot of live coding, I'm sort of out of ideas".\\

It's not even fair to say `out of ideas'. I had that cool idea about how I was going to finish it that I dogged both times. It's just exhausting. It really does take a lot of concentration.\\

I was done at the end and I was pretty happy it was done. I enjoyed it, I had a good time but I was glad it was done.\\
